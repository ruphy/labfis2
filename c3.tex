\chapter{Misura dell'accelerazione di gravità}
\section{Introduzione}
L'obiettivo dell'esperimento è misurare l'accelerazione gravitazionale $g$ attraverso due dei possibili metodi e confrontarli. 

\section{Pendolo di Kater}
\subsubsection{Introduzione}
\begin{wrapfigure}{R}{0.28\textwidth}
  \begin{center}
\includegraphics[scale=0.7]{../grafici/kater/kater.png}
  \end{center}
  \caption*{Il pendolo di Kater}
\end{wrapfigure}

Il pendolo di Kater è un pendolo fisico che può essere messo in oscillazione su due punti $c_1,\ c_2$ non coincidenti con il suo baricentro $G$. Il pendolo è costituito da un'asta rigida, due masse $m_a = 1400g$, $m_b=1000g$, e due coltelli, posizionati nei punti $c_1$ e $c_2$. La massa $m_a$ è mobile, e la sua distanza viene fatta variare rispetto a $c_1$. Dopodiché vengono misurati e riportati in tabella i periodi di oscillazione del pendolo, facendolo oscillare prima su $c_1$ e poi su $c_2$.

Il periodo del pendolo dipende ovviamente dalla posizione del baricentro rispetto al punto di sospensione. Chiamando $h_i$ la distanza del punto $i\in\{1,2\}$ dal baricentro:

$$ T_i = 2 \pi \sqrt{\frac{L_i}{g}} = 2 \pi \sqrt{\frac{I_i}{m gh_i}}$$

dove $L_i$ è la lunghezza ridotta del pendolo, $m$ è la massa totale e $I_i$ il momento d'inerzia del pendolo appeso per il punto $i$-esimo.
\\
Se si individuano i due punti nel quale il periodo $T_1 = T_2$ allora l'equazione si riduce a:

$$ T_0 = \ 2 \pi \sqrt{\frac{D}{g}}$$

La distanza $D= h_1 + h_2 $  è nota con grande precisione, e nel nostro caso è $D=99.4\ cm$. Misurando quindi il periodo $T_0=T_1=T_2$, possiamo calcolare $g$.
$$ g = \frac{4\pi^2D}{T^2}$$
\subsection{Procedimento}

In questo esperimento abbiamo utilizzato i seguenti strumenti:
\begin{itemize}
  \item Un metro a nastro, con sensibilità $1\ mm$
  \item Un cronometro digitale con fotocellula, che ha una precisione di $\pm 0.1 \cdot 10^{-4}\ s$
\end{itemize}

A questo punto abbiamo posto in oscillazione il pendolo dal coltello $c_1$ e misurato $T_1$; abbiamo poi ripetuto l'operazione per il coltello $c_2$, misurando $T_2$, e ripetuto la misura 10 volte al fine di stimare i migliori valori con i corrispondenti errori associati, mediante metodi statistici (media e deviazione standard).

Mantenendo fissa la massa $m_b$ ad una distanza $d = 15.4\ cm$ da $c_1$, allontaniamo o avviciniamo la massa $m_a$, e ripetiamo le misure.

La massa $m_a$ viene spostata per cercare una posizione in cui $T_1 \simeq T_2$, ovvero si è in una condizione per la quale $L\simeq h_1+h_2$. 
L'angolo di scostamento massimo $\theta_{max}$ che abbiamo usato è minore di $\frac{\pi}{36} \ rad$, ed è sufficientemente piccolo da permetterci di utilizzare la relazione 
$$ T = \ 2 \pi \sqrt{\frac{D}{g}}$$
invece che:
$$ T = \ 2 \pi \sqrt{\frac{D}{g}} \left( 1 + \frac{\theta^2}{16}\right)$$
Infatti, le due equazioni forniscono valori che differiscono solo per lo $0.04\% $.


\subsection{Dati}
Nella tabella sono raccolti tutti i periodi $T_1,T_2$ misurati in secondi. $d$ è la distanza tra il coltello $c_1$ e la massa $m_a$.\footnote{In tabella sono state riportate tutte le cifre significative che venivano lette dal cronometro}
\begin{center}
\begin{tabular}{*{8}{c}}
$d$& $6.9\ cm$ & $11.7\ cm$ & $13.5\ cm$  & $15.0\ cm$ & $17.9\ cm$ & $85.0\ cm$ & $81.9\ cm$ \\
\midrule
 \textbf{Coltello 1}&& & & & & \\
&2.1197 &2.0040&2.0279 &1.9809 & 1.9211 & 1.9934 & 1.9735\\
 &2.1125&2.0041 &2.0029&1.9796 & 1.9207 & 1.9932 & 1.9744 \\
 &2.1179&2.0043 &2.0021&1.9816 & 1.9264 & 1.9931 & 1.9746 \\
 &2.1204&2.0062 &2.0042&1.9794 & 1.9253 & 1.9932 & 1.9755 \\
 &2.1189&2.0060 &2.0026&1.9821 & 1.9113 & 1.9932 & 1.9765 \\
 &2.1129&2.0052&2.0012&1.9805 & 1.9278 & 1.9934 & 1.9769 \\
&2.1125 &2.0050&2.0014&1.9802 & 1.9278 & 1.9937 & 1.9770 \\
&2.1131 &2.0061&2.0008&1.9804 & 1.9277 & 1.9941 & 1.9760 \\
 &2.1143&2.0046&2.0036&1.9810 & 1.9278 & 1.9937 & 1.9754 \\
 &2.1131&2.0058&2.0005& 1.9831 & 1.9278 & 1.9935 & 1.9750 \\
 \midrule
$\bar{x}$& 2.1158 & 2.0085 & 2.0021 & 1.9806 & 1.9244 & 1.9935 & 1.9755\\
$\sigma$ & 0.0032 & 0.0019 & 0.0012 & 0.0102 & 0.0053 & 0.0003 & 0.0011\\
\midrule
\textbf{Coltello 2} && & & & & \\
&2.0279&2.0041&1.9983 &1.9934 & 1.9882 & 1.9946 & 1.9802 \\
&2.0284&2.0043&1.9983 &1.9946 & 1.9902 & 1.9939 & 1.9812 \\
&2.0289&2.0062&1.9967 &1.9911 & 1.9907 & 1.9943 & 1.9820 \\
&2.0289&2.0060&1.9978 &1.9919 & 1.9908 & 1.9941 & 1.9818 \\
&2.0288&2.0052&1.9978 &1.9940 & 1.9907 & 1.9940 & 1.9819 \\
&2.0288&2.0050&1.9986 &1.9923 & 1.9907 & 1.9952 & 1.9813 \\
&2.0288&2.0061&1.9986 &1.9939 & 1.9888 & 1.9954 & 1.9816 \\
&2.0288&2.0046&1.9988 &1.9934 & 1.9898 & 1.9943 & 1.9818 \\
&2.0286&2.0058&2.0001 &1.9914 & 1.9898 & 1.9947 & 1.9812  \\
&2.0286&2.0040&1.9994 &1.9958 & 1.9878 & 1.9941 & 1.9810 \\
 \midrule
$\bar{x}$& 2.0827 & 2.0053 & 1.9984 & 1.9932 & 1.9832 & 1.9945 & 1.9814\\
$\sigma$ & 0.0003 & 0.0009 & 0.0009 & 0.0015 & 0.0017 & 0.0005 & 0.0005\\
\bottomrule
\end{tabular}
\end{center}

\subsection{Analisi}
I periodi $T_1$ e $T_2$ sono funzione di $d$, essendo dipendenti dai momenti d'inerzia e della posizione del baricentro. I valori che ci interessano sono quelli in cui $T_1(d) = T_2(d)$.

Osserviamo che la $\sigma_T$ ha valori molto piccoli. Possiamo quindi ritenere trascurabile l'incertezza sul periodo ai fini delle nostre analisi. Si consideri infatti che il valore finale di $g$ sarà espresso con 3 cifre significative. Per facilitare successivamente la propagazione degli errori, consideriamo $T^2_{0}$ al posto di $T_0$
\\

L'equazione ha tre soluzioni, ma per semplicità, abbiamo deciso di concentrarci sulla prima, avendo cura di non considerare il caso banale $h_1=h_2$. Di seguito il grafico che rappresenza il quadrato del periodo in funzione della distanza $d$ ($T_2$ in verde, $T_1$ in nero).


\begin{center}
\includegraphics[scale=0.60]{../grafici/kater/kater-punti-raw.png}

\textit{La spezzata ha il solo scopo di collegare i punti al fine di rendere più leggibili i dati sperimentali.}
\end{center}

Per ricercare il miglior punto di intersezione minimizzando l'errore, abbiamo deciso di considerare le misurazioni dei periodi delle cinque distanze più vicine al valore che sembrava maggiormente plausibile (in questo caso abbiamo scelto 13.5 cm) e abbiamo calcolato, per ognuna di queste, le differenze tra i periodi $T_1$ e $T_2$.\\
\\
Scegliendo poi di considerare la posizione $d$ in funzione delle differenze di periodo, possiamo interpolare questi dati con una retta, e l'intercetta così trovata sarà il valore cercato di $d$ tale che $T_1 = T_2$. Poniamo sull'asse delle $y$ la distanza $d$, per poter calcolare l'incertezza $\sigma_y$ a posteriori, e sull'asse $x$ il periodo, la cui incertezza è trascurabile. 

\begin{center}
\includegraphics[scale=0.60]{../grafici/kater/intercetta.png}
\end{center}
Il valore così trovato è $d = 13.35 \pm 0.22\ cm$ \footnote{Vedere pagina successiva per spiegazioni}
\\
A questo punto, non potendo fare aggiustamenti estremamente precisi alla posizione delle masse, abbiamo preferito calcolare un valore plausibile per il periodo di oscillazione estapolandolo dai valori già trovati. Si è preferito interpolare con una retta i dati di  $T_2$, in quanto l'interpolazione è migliore rispetto a quella ottenibile usando i dati di $T_1$ ( $T_2$ ha infatti una $P(\tilde{\chi}^2 \geq \tilde{\chi_0}^2) \geq 11\%$).
In rosso il punto di intersezione trovato, $T^2_0$.

\begin{center}
\includegraphics[scale=0.70]{../grafici/kater/intersezione.png}
\end{center}

\subsection{Analisi delle incertezze}

Per calcolare le incertezza su $d$, abbiamo utilizzato l'equazione per il calcolo dell'incertezza a posteriori:
\begin{equation}
\sigma_y = \displaystyle\sqrt{\frac{\displaystyle\sum^n_{i=1}\left(y_i-q-mx_i \right)^2}{d}}
\end{equation}
e otteniamo $\sigma_d = 0.2246\ cm$. 
Propaghiamo l'errore così trovato su $g$:
$$\sigma_g = \frac{4 \pi^2}{T^2} \cdot \sigma_d = 0.022 \ m/s^2$$
$$ g = 9.807 \pm 0.022 \ m/s^2 $$
Considerando $g=9.81\ m/s^2$, ovvero la media dell'accelerazione di gravità sulla superficie terrestre:
$$\frac{g_{oss} - g_{teo}}{g_{teo}} \simeq 0.1\% $$



\section{Caduta libera}
\subsection{Procedimento}
L'apparato sperimentale per misurare l'accelerazione di gravità attraverso la caduta di gravi è un apparecchio piuttosto semplice. Questo consiste in:
\begin{itemize}
 \item Un'asta graduata verticale, sulla quale è possibile fissare la piattaforma di caduta della sferetta.
 \item Un bersaglio sulla quale cade la sferetta, collegato ad un cronometro
 \item Un cronometro sincronizzato con la piattaforma, che si attiva nel momento in cui la pallina si stacca dalla piattaforma e si ferma nel momento in cui la pallina colpisce un bersaglio.
\end{itemize}

Il cronometro ha una sensibilità di 1 ms, anche se possiamo stimare che probabilmente il suo errore è dell'ordine di 3-5 ms, dovuto principalmente al tempo di risposta. Questa stima è stata possibile a posteriori, attraverso l'analisi dei dati, mentre non si hanno a disposizioni le caratteristiche di costruzione dello strumento.

\subsection{Raccolta dati}
Tutti i valori riportati in tabella sono in millisecondi.
\begin{center}
\begin{tabular}{r|*{14}{c}}
\textbf{20 cm} & 197 & 199 & 197 & 200 & 199 & 203 & 200 & 203 & 196 & 199 & 196 & 199 & 197 & 205\\
& 198 & 199 & 198 & 197 & 199 & 198 & 197 & 198 & 200 & 198 & 199 & 199 & 198 & 204\\
\midrule
\textbf{35 cm} & 265 & 265 & 265 & 261 & 262 & 263 & 266 & 262 & 269 & 266 & 261 & 263 & 262 & 261\\
\midrule
\textbf{50 cm} & 314 & 315 & 314 & 321 & 319 & 316 & 315 & 315 & 315 & 314 & 314 & 315\\
\midrule
\textbf{60 cm} & 347 & 345 & 346 & 347 & 344 & 344 & 348 & 345 & 346 & 344 & 345 & 345\\
\midrule
\textbf{90 cm} & 424& 424& 424& 423& 425& 424& 426& 423& 424& 423& 422& 425& 423\\
\end{tabular}
\end{center}

\begin{center}
\includegraphics[scale=0.70]{../grafici/20cm.png}
$$\sigma_g = \left|\frac{-4y}{t^3}\right|\sigma_t = 0.23\  m/s^2$$
$$\sigma_{\bar{t}} = 0.430\ ms$$
$$\mathrm{Stima\ di\ g} = 10.10\ m/s^2$$
$$\mathrm{Stima\ (con\ corr.)\ di\ g} = 9.80\ m/s^2 $$
$$\frac{\Delta g_c}{g} = 0.00072$$
\includegraphics[scale=0.70]{../grafici/90cm.png}
$$\sigma_{\bar{t}} = 0.296\ ms $$
$$\mathrm{Stima\ di\ g} = 10.02\ m/s^2$$
$$\mathrm{Stima\ (con\ corr.)\ di\ g} = 9.88\ m/s^2 $$
$$\frac{\Delta g_c}{g} = 0.00707$$
\end{center}

La correzione applicata è di 3 ms aggiunti al tempo segnato dal cronometro. Non avendo dati più precisi sulla costruzione dell'apparato, assumiamo questo tempo come tempo di risposta dell'intero strumento. In tabella sono riportati i vari valori di $g$, $g_c$ e della distanza di questa misura dal valore vero (in percentuale). 

\begin{center}
\begin{tabular}{c|c|c|c|c}
$h$ (m) & $g$ (m/s$^2$) & $g_c$ (m/s$^2$) & $\Delta g/g$ & $\Delta g_c/g$\\
\midrule
0.20 & 10.10 & 9.80 & 0.02964 & 0.00072 \\
0.35 & 10.07 & 9.85 & 0.02659 & 0.00362 \\
0.50 & 10.04 & 9.85 & 0.02354 & 0.00435 \\
0.60 & 10.05 & 9.88 & 0.02475 & 0.00718 \\
0.90 & 10.02 & 9.88 & 0.02138 & 0.00707 \\
\end{tabular}
\end{center}
Calcoliamo il g medio con il metodo dei pesi: 
$$ g_{medio} = 9.85 \pm 0.25 \ m/s^2  $$
Nota: abbiamo trascurato la forza di attrito viscoso dell'aria e la forza di archimede poiché ininfluenti ai fini dell'esperienza.

\section{Conclusioni}

Il metodo di misura dell'accelerazione di gravità attraverso il pendolo di Kater si è rivelato essere molto più preciso di quello tramite caduta libera, nonostante le possibili incertezze introdotte da misure particolarmente incerte come la distanza della massa mobile dal primo coltello. Potendo effettuare le misure con strumenti più precisi, e riuscendo a raccogliere più misurazioni intorno al punto di intersezione tra $T_1$ e $T_2$, probabilmente si sarebbe riusciti a diminuire ulteriormente l'errore. 

Ciò nonostante, conoscendo bene l'apparato sperimentale, saremmo potuti arrivare a risultati simili con l'osservazione della caduta libera di un grave: con la correzione il valore misurato si discosta per un massimo dello 0.7\% dal valore vero.
