\chapter{Circuiti RC e RL (C3)}

\section{Corrente impulsata}
\subsubsection{Introduzione}

\subsection{Procedimento}

Oggetto di studio di questa esperienza è l'andamento della differenza di potenziale ai capi di resistenza e capacità o induttanza.
A tal fine costruiamo un circuito con i seguenti elementi:
[disegno]

\begin{itemize}
  \item Generatore di onde quadre.
  \item Un condensatore di capacità 367 nF più errore
  \item Un'induttore di induttanza sconosciuta
  \item Un oscilloscopio con due sonde
  \item Una resistenza di 667 Ohm (più errore)
\end{itemize}

Un'onda quadra viene generata e attraversa il condensatore. Prima che ciò avvenga, viene intercettata dalla sonda e visualizzata a schermo, per darci un riferimento sullo stato del circuito.
Con la seconda sonda, visualizziamo la forma dell'onda caratteristica della carica o della scarica di un condensatore.

Raccogliamo i dati (differenza di potenziale e tempo) dall'onda visualizzata sul display dell'oscilloscopio, tramite i cursori. Interpoliamo i dati raccolti con la curva caratteristica della carica/scarica di un condensatore.

Il valore stimato dall'interpolazione è $\tau=RC$ dell'equazione
$$V_R = \varepsilon e^{-t/\tau}$$

\subsection{Dati}
\subsubsection{Circuito RC}
\subsubsection{Circuito RL}

\subsection{Analisi}

\subsubsection{Circuito RC}
\subsubsection{Circuito RL}
\subsection{Analisi delle incertezze}


\section{Corrente alternata}
\subsection{Procedimento}
\subsection{Raccolta dati}
\subsubsection{Circuito RC}
\subsubsection{Circuito RL}

\section{Conclusioni}
\subsubsection{Circuito RC}
\subsubsection{Circuito RL}
