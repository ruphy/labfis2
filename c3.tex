\chapter{Circuiti RC e RL (C3)}

Oggetto di studio di questa esperienza è l'andamento della differenza di potenziale ai capi della resistenza e della capacità (o induttanza) di circuiti RC o RL in corrente impulsata e in corrente alternata.
A tal fine costruiamo un circuito con i seguenti elementi:
[disegno]

\begin{itemize}
  \item Generatore di onde
  \item Un condensatore di capacità 367 $nF$
  \item Un'induttore di induttanza sconosciuta, e resistenza 40 $\Omega$.
  \item Un oscilloscopio con due sonde
  \item Una resistenza di 667 $\pm 10 \Omega$  \footnote{Nel corso dell'esperimento abbiamo verificato, propagando l'errore associato alla resistenza, che esso risultava trascurabile. Pertanto d'ora in poi non se ne terrà più conto}
\end{itemize}

\section{Corrente impulsata}
\subsection{Procedimento}


Per simulare l'apertura e la chiusura del circuito impostiamo nel generatore la modalità onda quadra. Colleghiamo la prima sonda all'ingresso del circuito, subito prima del condensatore: in tal modo l'oscilloscopio confronta il segnale in ingresso con la messa a terra, e visualizza a schermo la differenza di potenziale, restituendo sul display l'onda quadra prodotta dal generatore.  
Analogamente una seconda sonda, posta ai capi della resistenza, visualizza la forma dell'onda caratteristica della carica o della scarica del condensatore/induttore. \\
Raccogliamo i dati campionando dei punti dall'onda visualizzata sul display dell'oscilloscopio, usando i cursori per ottenere la ddp corrispondete all'istante di tempo considerato (si fissa il primo cursore sullo zero e il secondo sul punto dell'onda di cui si vuole conoscere il potenziale: l'oscilloscopio visualizza la differenza tra il potenziale dei due cursori, e dunque la ddp desiderata).
Infine interpoliamo i dati raccolti con le curve caratteristiche della carica per i due differenti circuiti, ricavando il parametro $\tau$ (tempo caratteristico dei circuiti).

\subsubsection{Circuito RC}
Con un'onda quadra a 50 Hz, abbiamo raccolto i seguenti dati:

\begin{center}
\begin{tabular}{*{2}{c}}
Tempo ($\mu s$) & Ddp ($V$) \\
\midrule
0 & 18.20 \\
100 & 12.80 \\
200 & 9.00 \\
300 & 5.80 \\
400 & 4.40 \\
500 & 3.20 \\
600 & 2.20 \\
700 & 1.80 \\
800 & 1.20 \\
900 & 1.00 \\
\end{tabular}
\end{center}

Interpoliamo i dati raccolti con la curva della carica di un condensatore: 
$$V_R = \varepsilon e^{-t/\tau}$$


Il valore stimato dall'interpolazione è $\tau=281.7 \pm 5 \mu s$.

Valore atteso: $\tau=RC=248.5 \mu s$


Otteniamo un valore: $\chi^2 = 70 $. Dati 7 gradi di libertà, $\tilde{\chi}^2 = 10$.

\begin{center}
 \includegraphics[scale=0.70]{grafici/C3/fitcond.png}
\end{center}

Il valore stimato dall'interpolazione è $\tau=281.7 \pm 5 \mu s$.
Valore atteso: $\tau=RC=248.5 \mu s$

Per verificare l'accordo tra la legge e la distribuzione dei valori osservati operiamo il test del $\chi^2$:

$$ \chi^2 = \frac{\sum{(y_i - f(x_i,\tau )}}{\sigma_y^2} $$

Dividendo per i gradi di libertà (7 in questo caso), otteniamo un $\tilde{\chi}^2 = $ //
Leggiamo dai valori tabulati la probabilità percentuale di ottenere un valore di $\tilde{\chi}^2 \geq {\tilde{\chi}_0}^2 $ (dati i gradi di libertà). Otteniamo una probabilità del ... $\%$, maggiore della soglia di accettabilità, fissata al $5\%$. 


\subsubsection{Circuito RL}
Con un'onda quadra a 250 Hz, abbiamo raccolto i seguenti dati:

\begin{center}
\begin{tabular}{*{2}{c}}
Tempo ($\mu s$) & Ddp ($V$) \\
\midrule
5 & 5.00 \\
10 & 8.00 \\
15 & 10.60 \\
20 & 12.40 \\
25 & 13.60 \\
30 & 14.60 \\
35 & 15.40 \\
40 & 16.20 \\
45 & 16.40 \\
\end{tabular}
\end{center}
Interpoliamo i dati raccolti con la curva caratteristica della carica del circuito:

$$V_R = \varepsilon \left( 1-e^{-t/\tau} \right)$$

Il valore stimato dall'interpolazione è $\tau=16.05 \mu s$ \\
Non conoscendo a priori il valore di L non possiamo valutare l'accordo con il valore teorico: $\tau=\frac{L}{R}$.

I dati graficati:
\begin{center}
 \includegraphics[scale=0.70]{grafici/C3/fitindu.png}
\end{center}

Otteniamo un valore: $\chi^2 = 0,.38 $. Dati 7 gradi di libertà, $\tilde{\chi}^2 = 0.05$


\section{Corrente alternata}
\subsection{Procedimento}
Nella seconda parte dell'esperienza intendiamo misurare la risposta in frequenza (o funzione di trasferimento), definita come:

$H\left(\omega \right) = $  

Pertanto misuriamo la ddp ai capi di $R$ e $C$ o $L$ in modo analogo alle prima parte dell'esperienza, e la distanza tra due picchi delle onde visualizzate a schermo per determinare l'angolo $\phi$ di sfasamento ($\delta \phi = 2 \pi \delta t \ni$).

Per ricavare il rapporto $\frac{V_{R}}{V_{o}}$, dobbiamo ricavare $V_R$. Trovandoci in regime di corrente alternata, la leggge di Ohm è nella forma: $ V_o = Zi_o$ con $Z = R + jX$, impedenza del circuito.
Trattandosi di circuiti RC e RL in cui le impedenze sono collegate in serie, si ha $Z_{tot} = \sum Z_i$

\begin{itemize}
\item circuito RC $\rightarrow$ $Z=R-\frac{j}{\omega C}$
\item circuito RL $\rightarrow$ $Z=R+j\omega L$
\end{itemize}  

Allora: 

$$V_{Ro} = Ri_o = \frac{V_o}{Z} = \frac{RV_o}{\sqrt{R^2+X^2}} $$ 


\begin{itemize}
\item circuito RC $\rightarrow$ $X=\frac{1}{\omega C}$
\item circuito RL $\rightarrow$ $X=\omega L$
\end{itemize}

Mentre la fase risulta: 
$$\phi = \arctan \frac{X}{R} $$


\subsection{Circuito RC}


\begin{center}

\begin{tabular}{*{3}{c}}
Frequenza ($Hz$) & Delta V ($V_{out}/V_{in}$) & $\phi$ \\
\midrule
50 & 0.08 & 1.54\\
100 & 0.15 & 1.41\\
200 & 0.30 & 1.41\\
300 & 0.42 & 1.24\\
400 & 0.52 & 1.08 \\
1000 & 0.83 & 0.70\\
1500 & 0.90 & 0.49\\
2000 & 0.93 & 0.38\\
4000 & 0.97 & 0.20 \\
\end{tabular}
\end{center}

Lasciamo C come parametro libero, e interpoliamo i valori raccolti di $V_{out}$ e $V_{in}$ in funzione della frequenza:

$$\frac{V_{Ro}}{V_o} = \frac{R}{\sqrt{R^2+(\omega C)^{-2}}}$$

Dal fit otteniamo: $C=360 \cdot 10^{-9} \pm  F $ in ottimo accordo con il valore noto.

\begin{center}
 \includegraphics[scale=0.70]{grafici/C3/ddpcond.png}
\end{center}

\begin{center}
 \includegraphics[scale=0.70]{grafici/C3/fasecap.png}
\end{center}



Interpolo i dati della $\delta \phi$ utilizzando la funzione:

$$ \phi = \arctan \frac{1}{2\pi\nu C R} $$

con C parametro libero. Tramite il test del $\chi^2$ verifico l'accordo con i nostri dati, ottenendo un $\tilde{\chi}^2 = 1.05$

\subsection{Circuito RL}
\begin{center}

\begin{tabular}{*{3}{c}}
Frequenza ($Hz$) & Delta V ($V_{out}/V_{in}$) & $\phi (rad)$ \\
\midrule
1000& 0.47 & 0.19 \\
2000 & 0.62 & 0.35\\
3000 & 0.79 & 0.45\\
4000 & 1.05 & 0.58\\
6000 & 1.48 & 0.79\\
8000 & 2.01 & 1.01\\
10000 & 2.56 & 0.94\\
15000 & 3.64 & 1.13\\
20000 & 4.28 & 1.26\\
30000 & 5.08 & 1.36\\
50000 & 5.89 & 1.57\\
\end{tabular}
\end{center}


\begin{center}
 \includegraphics[scale=0.70]{grafici/C3/ddpindu.png}
\end{center}

Anche questa volta interpoliamo i dati raccolti con la funzione:

$$\frac{V_{Ro}}{V_o} = \frac{R}{\sqrt{R^2+(\omega L)^2}}$$

La stima di L risulta in ottimo accordo con quella realizzata nella prima parte dell'esperienza.
$L=0.0012 \pm H $

Il test del $\chi^2$ conferma la validità delle ipotesi fatte.

(chi quadro della delta phi 0.96)


Interpolo i dati utilizzando la funzione:

$$ \phi = \arctan \frac{2\pi\nu L}{R} $$

dove L, è il parametro libero da stimare.


\begin{center}
 \includegraphics[scale=0.70]{grafici/C3/faseindu.png}
\end{center}

Da questo fit, otteniamo una $L=0.0017$ H, e un $\chi^2 = 8.48 $, considerando una $\sigma_{y_i} = 2 \pi \nu_i \sigma_{t}$, dove  per $\sigma_t = \pm 2 \mu s$. L'incertezza sulla frequenza è trascurabile, e per questo motivo abbiamo propagato l'incertezza solo su $\sigma_t$. Poichè $\sigma_{y_i}$ è proporzionale alla frequenza, l'incertezza della misura di $\phi$ sulle frequenze più elevate diviene dunque rilevante rendendo possibile accettare un $\chi^2$ così piccolo, pur per un fit che, graficamente, non sembra in buon accordo. Il grafico è inoltre logaritmico, e accentua questo effetto
 

\section{Conclusioni}


\begin{center}
\begin{tabular}{*{2}{c}}
Parametro & $\chi^2$ \\
\midrule
Condensatore $\tau =$ & 0.2 \\
Induttore $\tau =$ & 0.2 \\

\end{tabular}
\end{center}

