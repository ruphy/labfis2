\begin{sagesilent}
import numpy as np
import matplotlib.pyplot as plt

rc = np.recfromcsv("dati/C2-RC.csv")
rl = np.recfromcsv("dati/C2-RL.csv")



def stampa_dati(wa, header):
  s = r"\begin{tabular}{c*{" + "%d" % (len(wa.dtype)-1)
  s += r"}{|c}}"
  s += "%s \\\\" % (header)
  s += r"\midrule"
  for i in range(0, len(wa)):
    a = ["%s" %x for x in wa[i]]
    s += "%s \\\\" % join(a, "&")
  s += r"\end{tabular}"
  return s
  
  

\end{sagesilent}



\chapter{C2}


\section{Risposta in frequenza del multimetro}



\begin{center}
\includegraphics[scale=0.75]{grafici/C2/cv.png} 
\end{center}

Per il grafico in alto,che mostra la lettura dell'intensità di corrente in funzione della frequenza, l'errore è dato dalla sensibilità dello strumento: $\sigma_i = 0.1\ mA$.
\

Il secondo grafico, che mostra i limiti operativi del multimetro rispetto la frequenza, L'errore sulla lettura dal multimetro è la sensibilità dello strumento: $\sigma_{mu} = 0.005\ V$. Per stimare l'errore su $V_{RM}$ abbiamo usato la deviazione standard, assumendo quindi che $V_{RMS}$ dovrebbe rimanere costante. 
Indi per cui, la propagazione degli errori risulta:
$$\sigma_r = \sqrt{\frac{\sigma_{mu}^2}{V_{RMS}^2} + \frac{\sigma_{rms}^2}{V_{RMS}^4}}$$

%E' giusto pensare che rimanga costante?f'

In nostro multimetro opera nella banda di $True\ RMS$, fino a  $1000\ Hz$.   

%Quesiti:
%
%
%
%
%

\section{Misura di impedenze ignote}

Scopo di questa seconda parte è misurare l'andamento dell''impedenza di un circuito RC e RL in funzionde della frequenza alternata.

Gli errori su queste misure è l'incertezza dell'oscilloscopio: ($\pm$) la sua risoluzione, propagati secondo la seguente formula.

$$\sigma_r = \sqrt{\frac{\sigma_{deltaV}^2}{(2*i)^2} + \frac{\sigma_{i}^2}{(2*i)^4}}$$


Utilizziamo la seguente funzione per propagare i dati:
$Z = \frac{1}{\omega \cdot C}$

\begin{center}
\includegraphics[scale=0.75]{grafici/C2/rc.png} 
\end{center}
Parametri ricavati dal fit:
\
$C = 374.53\pm 8.44 nF$.
\
$\chi_{rid}^2= 1.4461 $ 


Utilizziamo la seguente funzione per propagare i dati:
$Z = w\cdot L$

\begin{center}
\includegraphics[scale=0.75]{grafici/C2/rl.png} 
\end{center}

$L = 0.016074\pm 0.000092 H$ \
$\chi_{rid}^2 = 1.4546$ 




