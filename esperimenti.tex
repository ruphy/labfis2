\documentclass[a4paper,10pt]{report}
\usepackage[cm]{fullpage}
\usepackage[utf8]{inputenc}
\usepackage{amsmath}
\usepackage{amsthm}
\usepackage{amssymb}
\usepackage{appendix}
\usepackage{booktabs}
\usepackage[table]{xcolor}
\usepackage{amssymb}
\usepackage{multirow}
\usepackage{fullpage}
\usepackage{float}
\usepackage{wrapfig}
\usepackage{subfig}
\usepackage{graphicx}
\usepackage{listings}
\usepackage{color}
\usepackage{textcomp}
\usepackage{sagetex}


\usepackage{hyperref}

\definecolor{listinggray}{gray}{0.9}
%\definecolor{lbcolor}{rgb}{0.9,0.9,0.9}

\addtolength{\voffset}{-10pt}

\hypersetup{colorlinks=false}
 
\lstset{
%	backgroundcolor=\color{lbcolor},
	tabsize=4,
%	rulecolor=,
	language=python,
        basicstyle=\scriptsize,
        upquote=true,
        aboveskip={1.5\baselineskip},
        columns=fixed,
        showstringspaces=false,
        extendedchars=true,
        breaklines=true,
        prebreak = \raisebox{0ex}[0ex][0ex]{\ensuremath{\hookleftarrow}},
%        frame=single,
        showtabs=false,
        showspaces=false,
        showstringspaces=false,
        identifierstyle=\ttfamily,
        keywordstyle=\color[rgb]{0,0,1},
        commentstyle=\color[rgb]{0.133,0.545,0.133},
        stringstyle=\color[rgb]{0.627,0.126,0.941},
} \hypersetup{colorlinks=false}

\renewcommand\chaptername{Esperimento}
 
\DeclareGraphicsExtensions{.pdf,.png,.jpg}


\author{Marco Giglio, Maria Cristina Fortuna, Riccardo Iaconelli}
% Title Page
\title{Relazioni dal Laboratorio di Fisica 2}

\begin{document}

\maketitle

\tableofcontents

%INIZIALIZZAZIONE e LIBRERIE
\begin{sagesilent}
#Importo librerie necessarie
import numpy as np
from scipy import odr
import matplotlib.pyplot as plt
import matplotlib.mlab as ml
from scipy.special import chdtrc
import numpy.lib.recfunctions as rf

#classica funzione chiquadrato, non cancellare
def chiquad(xdata, ydata, yfunc, ysigma = (), param=()):
    yteo = yfunc(xdata, *param)
    ddof = len(xdata) - 1 - len(param)
    if (np.isscalar(ysigma)):
        ys = np.ones_like(xdata)*ysigma
    else:
        ys = ysigma
 
    if (len(ys)):
        return sum(((ydata-yteo)/ys)**2)
    else:
        return sum((ydata-yteo)**2/yteo)

def fit_chiquad(xdata, ydata, yfunc, init_guess, ysigma = ()):
    # Fit!
    mymodel = odr.Model(yfunc)
    mydata = odr.RealData(xdata, ydata)
    myodr = odr.ODR(mydata, mymodel, beta0=init_guess, maxit=5000)
    myout = myodr.run()
    
    # Chi^2
    yteo = yfunc(myout.beta, xdata)
    ddof = len(xdata) - 1 - len(myout.beta)

    if (np.isscalar(ysigma)):
        ys = np.ones_like(xdata)*ysigma
    else:
        ys = ysigma
 
    if (len(ys)):
        chiquad = sum(((ydata-yteo)/ys)**2)
    else:
        chiquad = sum((ydata-yteo)**2/yteo)

    prob = chdtrc(ddof,float(chiquad))
    
    return (myout, chiquad, prob)
        
#Funzione stampadati
def stampa_dati(datiarr, header):
  s = r"\begin{tabular}{c*{" + "%d" % (len(datiarr.dtype)-1)
  s += r"}{|c}}"
  s += "%s \\\\" % (header)
  s += r"\midrule"
  for i in range(0, len(datiarr)):
    a = ["%.5G" %x for x in datiarr[i]]
    s += "%s \\\\" % join(a, "&")
  s += r"\end{tabular}"
  return s
        
\end{sagesilent}


%\begin{sagesilent}
import numpy as np

def stampa_dati(wa, header):
  s = r"\begin{tabular}{c*{" + "%d" % (len(wa.dtype)-1)
  s += r"}{|c}}"
  s += "%s \\\\" % (header)
  s += r"\midrule"
  for i in range(0, len(wa)):
    a = ["%s" %x for x in wa[i]]
    s += "%s \\\\" % join(a, "&")
  s += r"\end{tabular}"
  return s
\end{sagesilent}



\chapter{C1}

L'obiettivo del nostro esperimento è misurare la validità della legge di Ohm per varie configurazioni di un circuito. 

Al fine di misurare corrente e potenziale, colleghiamo al nostro circuito due multimetri digitali. Il primo, che ha la funzione di voltmetro, lo poniamo ai capi della nostra resistenza collegato in parallelo; il secondo, in modalità amperometro, è posto in serie subito dopo la resistenza. 

Di seguito, gli strumenti con la loro precisione:
\begin{itemize}
\item Voltmetro (multimetro portatile)
\item Amperometro (multimetro da banco)
\item Generatore da banco, resistenza interna ignota.
\end{itemize}


\section{Resistenze}

Ricaviamo, tramite  il fit della funzione $V=R*I$ dove R è parametro da stimare, il valore di due resistenze ignote. Di seguito i dati del circuito 1 e del circuito 2. 


\begin{center}
\begin{tabular}{*{4}{c}}
Corrente1($nA$) & Potenziale1($mV$) & Corrente2($nA$) & Potenziale2($mV$)\\
\midrule
396 & 201 & 3043 & 207\\
512 & 261 & 3667 & 250\\
706 & 360 & 4703 & 319\\
890 & 454 & 6365 & 432\\
1126 & 574 & 8984 & 609\\
1242 & 633 & 12326 & 836\\
1557 & 794 & 16898 & 1145\\
1812 & 925 & 340 & 24\\
1971 & 1005 & 725 & 50\\
2290 & 1168 & 966 & 66\\
2524 & 1286 & 1595 & 108\\
2850 & 1454 & 1822 & 124\\
3116 & 1589 & 2160 & 146\\
3407 & 1737 & 2302 & 157\\
3883 & 1980 & 2584 & 176\\
4229 & 2157 & 3100 & 211\\
4598 & 2344 & 3370 & 229\\
5072 & 2586 & 4036 & 274\\
5505 & 2807 & 4348 & 295\\
5820 & 2967 & 4596 & 312\\
6257 & 3191 & 4894 & 332\\
6738 & 3436 & 5578 & 379\\
7521 & 3835 & 6613 & 449\\
7880 & 4018 & 6963 & 473\\
8493 & 4331 & 7371 & 500\\
8852 & 4514 & 7864 & 533\\
9135 & 4658 & 8603 & 584\\
9431 & 4809 & 9066 & 615\\
9720 & 4957 & 9667 & 656\\
9972 & 5085 & 10816 & 733\\

\end{tabular}
\end{center}


\includegraphics[scale=0.75]{grafici/C1/res1.png}
\

Parametri trovati dal fit: \
$R= 5099 \pm 0.3 \ \Omega$

$q = -0.113 \pm 0.147\ mV$


\includegraphics[scale=0.75]{grafici/C1/res2.png}

$R = 677 \pm 2.02\cdot 10^{-1} \Omega$

$q = 0.72 \pm 1.35\cdot 10^{-1} \ mV $
\

\section{Serie e Parallelo}

L'incertezza sulla misura è quello dello strumento. Nel caso dell'amperometro $\sigma_a = 0.0001\ A$ , per il voltmetro
$sigma_v = 0.001\ V$ per le misure in serie, e $sigma_v = 0.01\ V$ per le misure in parallelo.
\\

\begin{sagesilent}
 
 import numpy as np
 import matplotlib.pyplot as plt
 from scipy import odr

 cs = np.recfromcsv("dati/c1-serie.csv",delimiter="\t")
 
 cp = np.recfromcsv("dati/c1-parallelo.csv",delimiter="\t")

 #Errori presi da risoluzione strumento
 yerr1=0.001
 yerr2=0.01
 xerr=0.0001
 
 def fu(P,x):
    return P[0]*x + P[1]

 mymod = odr.Model(fu)
 mydata = odr.RealData(cs['i'], cs['v'], sy=yerr1)
 myodr = odr.ODR(mydata, mymod, beta0=[50, 0.1], maxit=1000)
 myout = myodr.run()
 myout.pprint()
 
 mymod = odr.Model(fu)
 mydata = odr.RealData(cp['i'], cp['v'], sy=yerr1)
 myodr = odr.ODR(mydata, mymod, beta0=[50, 0.1], maxit=1000)
 myout = myodr.run()
 myout.pprint()

 plt.clf()
 plt.subplot(211)
 plt.errorbar(cs['i'],cs['v'],yerr1,xerr,fmt='b.')
 plt.ylabel("Potenziale V")
 plt.grid(True)
 
 plt.subplot(212)
 plt.errorbar(cp['i'],cp['v'],yerr2,xerr,fmt='r.')
 plt.grid(True)
 plt.ylabel("Potenziale $V$")
 plt.xlabel("Corrente $A$")
 plt.savefig("grafici/C1/c1B.png",dpi=300)
 
\end{sagesilent}


\sagestr{stampa_dati(cs, r"Potenziale (V) & Corrente (A)")}
\

\sagestr{stampa_dati(cp, r"Potenziale (V) & Corrente (A)")}

\includegraphics[scale=0.75]{grafici/C1/c1B.png}

Il primo grafico mostra l'andamento del potenziale in funzione della corrente per un circuito dotato di due resistenze in serie da $1297\ \Omega$
Dal fit ricaviamo un valore di $2117 \Omega$, che ben si avvicina al valore teorico $R_{eq}=R_1+R_2= 2594\ \Omega$.

Il secondo grafico, sono invece le due medesime resistenze in parallelo. Dal fit si ricava una resistenza equivalente di
$R = 604 \Omega$, che ben si accorda con il valore teorico $R_{eq} = \frac{R_1 \cdot R_2}{R1+R2}=648\ \Omega$.

Gli errori sui fit non sono stati riportati perchè trascurabili. 
\\

\section{Resistenza interna dei multimetri}
Misuro la resistenza interna del voltometro, mantenendo costante la ddp a $14.5\ V$ e variando la resistenza all'interno del circuito. 
Il voltmetro è in parallelo al circuito, perciò $R_i$:

$$R_i = \frac{RV}{RI-V} $$

dove R è la resistenza variabile, I la corrente nel circuito e $V= 14.5\ V$

\begin{center}
\begin{tabular}{*{2}{c}}
Resistenza $M\Omega$ & Corrente $nA$\\
\midrule
7&      36\\
9&      32\\
10& 30\\
11&     29\\
12&     28\\
13&     27\\
14&     26\\
15&     26\\
16&     25\\
17&     24\\

\end{tabular}

\end{center}
La resistenza risulta $9.20 \pm 0.27 \ M \Omega$.


Misuriamo la resistenza interna dell'amperometro, che è collegato in serie al circuito. 

$$R_i = \frac{V-RI}{I}$$

In questo caso, R è fissato ($R=0.5 \Omega$) e sono V e I a variare
\begin{center}
\begin{tabular}{*{2}{c}}
Volt $mV$ & Corrente $nA$\\
\midrule
22&      1769\\
40&      3271\\
60&      5047\\
105&     8938\\
133&     11201\\
142&     12021\\
164&     13882\\
174&     14745\\
208&     17604\\
232&     19671\\



\end{tabular}
\end{center}

La resistenza interna risulta $11.42 \pm 0.45 \Omega$

Colleghiamo una piccola lampada a filamento al circuito, e verifichiamo che il suo comportamento resistivo non segue la legge di Ohm. 

\includegraphics[scale=0.75]{grafici/C1/lampa.png}

La lampadina ha un comportamento non-ohmico nel momento in cui il filamento si scalda sufficientemente e inizia ad emettere luce ($500mV$).


\section{Partitore resistivo}
Il partitore di corrente è un circuito utilizzato per calcolare la corrente elettrica che scorre attraverso un'impedenza o attraverso un circuito quando esso viene connesso in parallelo con un'altra impedenza.
Nel nostro caso, stiamo analizzando un partitore resistivo. 

\subsection{Situazione senza carico}
\begin{center}
\begin{tabular}{*{2}{c}}
$V_{in}$ & $\frac{V_{in}}{V_{out}}$\\
\midrule
329.0 & 0.5015 \\
493.0 & 0.501 \\
544.0 & 0.5018 \\
618.0 & 0.5016 \\
667.0 & 0.5007 \\
776.0 & 0.5 \\
803.0 & 0.5006 \\
927.0 & 0.5005 \\
1078.0 & 0.5 \\
1285.0 & 0.4996 \\
\end{tabular}

\end{center}


\includegraphics[scale=0.75]{grafici/C1/part11.png}

\subsection{Situazione con carico}
\begin{center}

\begin{tabular}{*{2}{c}}
$V_{in}$ & $\frac{V_{in}}{V_{out}}$\\
\midrule
220.0 & 3.1 \\
276.0 & 3.1051 \\
302.0 & 3.1126 \\
410.0 & 3.1073 \\
511.0 & 3.1115 \\
559.0 & 3.1055 \\
624.0 & 3.109 \\
752.0 & 3.109 \\
906.0 & 3.1093 \\
1036.0 & 3.111 \\
\end{tabular}

\end{center}
\includegraphics[scale=0.75]{grafici/C1/part22.png}


\begin{center}
\begin{tabular}{*{4}{c}}
Corrente1($nA$) & Potenziale1($mV$) & Corrente2($nA$) & Potenziale2($mV$)\\
\midrule
2.378 & 201 & 3043 & 207\\
2.500 & 261 & 3667 & 250\\
2.972 & 360 & 4703 & 319\\
3.290 & 454 & 6365 & 432\\
4.262 & 574 & 8984 & 609\\
4.856 & 633 & 12326 & 836\\
5.640 & 794 & 16898 & 1145\\
5.090 & 925 & 340 & 24\\
1971 & 1005 & 725 & 50\\
2290 & 1168 & 966 & 66\\
2524 & 1286 & 1595 & 108\\
2850 & 1454 & 1822 & 124\\
3116 & 1589 & 2160 & 146\\
3407 & 1737 & 2302 & 157\\
3883 & 1980 & 2584 & 176\\
4229 & 2157 & 3100 & 211\\
4598 & 2344 & 3370 & 229\\
5072 & 2586 & 4036 & 274\\
5505 & 2807 & 4348 & 295\\
5820 & 2967 & 4596 & 312\\
6257 & 3191 & 4894 & 332\\
6738 & 3436 & 5578 & 379\\
7521 & 3835 & 6613 & 449\\
7880 & 4018 & 6963 & 473\\
8493 & 4331 & 7371 & 500\\
8852 & 4514 & 7864 & 533\\
9135 & 4658 & 8603 & 584\\
9431 & 4809 & 9066 & 615\\
9720 & 4957 & 9667 & 656\\
9972 & 5085 & 10816 & 733\\

\end{tabular}
\end{center}



%\begin{sagesilent}
import numpy as np

rc = np.recfromcsv("dati/C2-RC.csv")
rl = np.recfromcsv("dati/C2-RL.csv")


def stampa_dati(wa, header):
  s = r"\begin{tabular}{c*{" + "%d" % (len(wa.dtype)-1)
  s += r"}{|c}}"
  s += "%s \\\\" % (header)
  s += r"\midrule"
  for i in range(0, len(wa)):
    a = ["%s" %x for x in wa[i]]
    s += "%s \\\\" % join(a, "&")
  s += r"\end{tabular}"
  return s
\end{sagesilent}



\chapter{C2}

\section{Risposta in frequenza del multimetro}



\begin{center}
\includegraphics[scale=0.75]{grafici/C2/cv.png} 
\end{center}

Per il grafico in alto,che mostra la lettura dell'intensità di corrente in funzione della frequenza, l'errore è dato dalla sensibilità dello strumento: $\sigma_i = 0.1\ mA$.
\

Il secondo grafico, che mostra i limiti operativi del multimetro rispetto la frequenza, L'errore sulla lettura dal multimetro è la sensibilità dello strumento: $\sigma_{mu} = 0.005\ V$. Per stimare l'errore su $V_{RM}$ abbiamo usato la deviazione standard, assumendo quindi che $V_{RMS}$ dovrebbe rimanere costante. 
Indi per cui, la propagazione degli errori risulta:
$$\sigma_r = \sqrt{\frac{\sigma_{mu}^2}{V_{RMS}^2} + \frac{\sigma_{rms}^2}{V_{RMS}^4}}$$

%E' giusto pensare che rimanga costante?f'

\section{Misura di impedenze ignote}

Scopo di questa seconda parte è misurare l'impedenza di un circuito RC e RL in frequenza alternata. Mostreremo la dipendenza di Z dalla frequenza.


\begin{center}

%\sagestr{stampa_dati(rc, r"Frequenza (Hz) & I (mA) & Valore mu & $V_{rms}$ (V)")}
\end{center}


\begin{center}
\includegraphics[scale=0.75]{grafici/C2/rc.png} 
\end{center}

Dal fit ricavo: $C = 374.53\pm8.44 nF$, che rientra nei valori aspettabili, dato che il valore teorico è $367 nF$. 
Il $chi^{\tilde}^2= 1.4461 $ 

%chi quadro rotto da sistemare.

\begin{center}

%\sagestr{stampa_dati(rl, r"Frequenza (Hz) & I (mA) & Valore mu & $V_{rms}$ (V)")}
\end{center}



\begin{center}
\includegraphics[scale=0.75]{grafici/C2/rl.png} 
\end{center}

$L = 0.016074\pm 0.000092 H$
$\chi^{\tilde}^2 = 1.4546$





% \chapter{Circuiti RC e RL (C3)}

Oggetto di studio di questa esperienza è l'andamento della differenza di potenziale ai capi di resistenza e capacità o induttanza.
A tal fine costruiamo un circuito con i seguenti elementi:
[disegno]

\begin{itemize}
  \item Generatore di onde
  \item Un condensatore di capacità 367 nF più errore
  \item Un'induttore di induttanza sconosciuta
  \item Un oscilloscopio con due sonde
  \item Una resistenza di 667 Ohm (più errore)
\end{itemize}

\section{Corrente impulsata}
\subsection{Procedimento}


Per simulare l'apertura e la chiusura del circuito impostiamo nel generatore la modalità onda quadra. Una sonda posta prima del condensatore mostra sullo schermo dell'oscilloscopio il segnale.  
Una seconda sonda, posta ai capi della resistenza, visualizza la forma dell'onda caratteristica della carica o della scarica di un condensatore.

Raccogliamo i dati (differenza di potenziale e tempo) dall'onda visualizzata sul display dell'oscilloscopio, tramite i cursori. Interpoliamo i dati raccolti con la curva caratteristica della carica/scarica di un condensatore.

Il valore stimato dall'interpolazione è $\tau=RC$ dell'equazione
$$V_R = \varepsilon e^{-t/\tau}$$

\subsection{Dati}

\subsubsection{Circuito RC}

Per il circuito  RC in carica, abbiamo raccolto i seguenti dati. 

\begin{center}
 \includegraphics[scale=0.70]{grafici/C3/fitcond.png}
\end{center}


Con un'onda quadra a 50 Hz, abbiamo raccolto questi dati:
\begin{center}
\begin{tabular}{*{2}{c}}
Tempo ($\mu s$) & Ddp ($V$) \\
\midrule
0 & 18.20 \\
100 & 12.80 \\
200 & 9.00 \\
300 & 5.80 \\
400 & 4.40 \\
500 & 3.20 \\
600 & 2.20 \\
700 & 1.80 \\
800 & 1.20 \\
900 & 1.00 \\
\end{tabular}
\end{center}


\subsubsection{Circuito RL}
Con un'onda quadra a 250 Hz, abbiamo raccolto questi dati:

\begin{center}
 \includegraphics[scale=0.70]{grafici/C3/fitindu.png}
\end{center}

\begin{center}
\begin{tabular}{*{2}{c}}
Tempo ($\mu s$) & Ddp ($V$) \\
\midrule
5 & 5.00 \\
10 & 8.00 \\
15 & 10.60 \\
20 & 12.40 \\
25 & 13.60 \\
30 & 14.60 \\
35 & 15.40 \\
40 & 16.20 \\
45 & 16.40 \\
\end{tabular}
\end{center}

\subsection{Analisi}

\subsubsection{Circuito RC}
\subsubsection{Circuito RL}
\subsection{Analisi delle incertezze}


\section{Corrente alternata}
\subsection{Procedimento}


Vogliamo misurare la risposta in frequenza 
\subsection{Raccolta dati}

\subsubsection{Circuito RC}

\begin{center}
 \includegraphics[scale=0.70]{grafici/C3/ddpcond.png}
\end{center}

\begin{center}
\begin{tabular}{*{2}{c}}
Frequenza ($Hz$) & Delta V ($V_{out}/V_{in}$) \\
\midrule
50 & 0.08 \\
100 & 0.15 \\
200 & 0.30 \\
300 & 0.42 \\
400 & 0.52 \\
1000 & 0.83 \\
1500 & 0.90 \\
2000 & 0.93 \\
4000 & 0.97 \\
\end{tabular}
\end{center}


\subsubsection{Circuito RL}

\begin{center}
 \includegraphics[scale=0.70]{grafici/C3/ddpindu.png}
\end{center}

\begin{center}
\begin{tabular}{*{2}{c}}
Frequenza ($Hz$) & Delta V ($V_{out}/V_{in}$) \\
\midrule
50 & 0.08 \\
100 & 0.15 \\
200 & 0.30 \\
300 & 0.42 \\
400 & 0.52 \\
1000 & 0.83 \\
1500 & 0.90 \\
2000 & 0.93 \\
4000 & 0.97 \\
\end{tabular}
\end{center}

\section{Conclusioni}
\subsubsection{Circuito RC}
\subsubsection{Circuito RL}

\chapter{C4}

\begin{sagesilent}

import numpy as np
from scipy import odr
import matplotlib.pyplot as plt

trasfm = np.recfromcsv("dati/C4-tensione-fase.csv")
chiquad = load("sobj/chiquad.sobj")
#Sottosmorzamento tempo-volt
#Smorzamento critico tempo-volt
#Sovrasmorzamento tempo-volt
\end{sagesilent}


\begin{sagesilent}

#Funzione di trasferimento del modulo: freq-volt
#RLC in corrente alternata
dati = np.recfromcsv("dati/C4-tensione-fase.csv")
var('x,l,c,v,w')
r = 300
def f(x, l, w, v):
    return (v*2*3.14*x*w)/( sqrt( (2*3.14*x*w)^2+(w*l/r)^2*((2*3.14*x)^2-w^2)^2 ) )
    
puls = dati['frequenza_hz'] #*2*3.14
ennupla = list(((puls[i]), dati['v_out__v_in'][i]) for i in range(0,len(dati['frequenza_hz'])))


fit = find_fit(ennupla, f, parameters=[l,w,v], variables=[x], solution_dict=True)

print fit

plt.clf()
xin = np.arange(100, 100000, 10)
yin = f(xin, fit[l], fit[w], fit[v])
plt.semilogx(xin,yin, 'g--')
plt.semilogx(dati['frequenza_hz'], dati['v_out__v_in'], 'ok')

plt.xlabel(r"$\omega$ [$rad/s$]")
plt.ylabel(r"$V(\omega)$ [$rad$]")
plt.grid(True)

plt.savefig("grafici/C4-ris.png", dpi=300)

#Funzione di trasferimento fase:freq-rad
  
\end{sagesilent}

\includegraphics[scale=0.75]{grafici/C4-ris.png}



%Sottosmorzamento
\begin{sagesilent}
 
om = 2*n(pi)*100
t =np.array([190,570,950,1520])
picchi=np.array([520,-148,60,-24])

def func(P,x):
    return P[0]*P[1]*P[2]*exp(-P[3]*t)*(P[3]*cos(om*x)+om*sin(100*x))

#var('x,r,c,v,g')
#def f(x,r,c, v, g):
#    return r*c*v*exp(-g*t)*(g*cos(om*x)+om*sin(100*x))
    
plt.clf()
plt.plot(t,picchi,'bo')

mod = odr.Model(func)
data = odr.RealData(t,picchi)
done = odr.ODR(data, mod, beta0=[1.,1.,1.,1.],maxit=1000)
sotto = done.run()

#xin = np.arange(min(t),max(t),1)
#yin = func(sotto.beta, xin)

#plt.plot(xin,yin,'g--')
plt.xlabel(r"$\omega$ [$rad/s$]")
plt.ylabel(r"$V(\omega)$ [$rad$]")
plt.savefig("grafici/C4-om.png",dpi=300)
 
\end{sagesilent}

\begin{center}
 \includegraphics[scale=0.75]{grafici/C4-om.png}
\end{center}

%Funzione di trasferimento del modulo
% \begin{sagesilent}
% 
% freq = trasfm['frequenza_hz']
% v = trasfm['v_out_v_in']
% 
% #Funzione di trasferimento del modulo
% 
% def fu(P,x):
%     return (P[0]*2*3.14*x*P[1])/( sqrt( (2*3.14*x*P[1])^2+(w*P[2]/R)^2*((2*3.14*x)^2-[1]^2)^2 ) )
% 
%   
% myodr, chi = fit_chiquad(freq, v, fu, ysigma=yerr, param=[mytransm.beta])
% print "Chi quadro %.4f" % chi
%   
% \end{sagesilent}


%\chapter{ Spettrometro (O3)}


\section*{Reticolo}

\subparagraph{Introduzione}

Nella prima parte dell'esperienza misuriamo le lunghezze d'onda corrispondenti a differenti righe spettrali emesse da una sorgente di gas eccitato, attraverso l'uso di un reticolo a diffrazione. \\
Il reticolo a diffrazione consiste in una lastra di vetro su cui sono tracciate delle linee parallele molto sottili. La distanza tra le fenditure dà il passo del reticolo.\\
La luce proveniente dalla sorgente viene fatta passare prima attraverso una fenditura, e successivamente attraverso una lente convergente, il cui piano focale coincide proprio con il piano su cui giace la fenditura: tale sistema prende il nome di collimatore. I raggi paralleli uscenti dalla lente intercettano il reticolo, che attraverso una procedura sperimentale viene posto ortogonalmente al fascio, e da lì vengono deviati di un angolo $\theta$. Un telescopio montato su un piatto rotante permette di osservare le righe d'emissione del gas, ruotando opportunamente la piattaforma secondo la direzione $\theta_{\lambda} $ data dalla legge:
\begin{equation}
sin \theta_{\lambda} = m \frac{\lambda}{d}
\label{eq:theta}
\end{equation}


Da tale legge, dunque possiamo risalire alla lunghezza d'onda, attraverso la lettura dell'angolo di cui si è ruotato il telescopio (la piattaforma è dotata di un goniometro).

\subparagraph{Passo del reticolo}

Prima di iniziare la raccolta dati è necessario posizionare il reticolo in modo tale che risulti perpendicolare al fascio di luce incidente. A tal fine misuriamo gli angoli di deviazione per massimi dello stesso ordine, e lo zero corrisponde alla direzione della luce non deviata. Otteniamo $\theta_{0} = 283 3' $ \\
Per stabilire il passo del reticolo utilizziamo una sorgente di cui è nota la lunghezza d'onda (lampada a sodio: $\lambda_{Na} = 589.3 $ nm ), e giriamo la \ref{eq:theta} per ricavare d. In particolare utilizziamo una versione della \ref{eq:theta} che diminuisca l'errore sperimentale a cui sono soggette le misure, in cui a $\theta$ si sostituisce la media dei due angoli destro e sinistro $\theta_{dx} $ e $\theta_{sx}$.

\begin{equation}
d = \frac{n \lambda}{sin(\frac{\theta_{dx}-\theta_{sx}}{2})}
\end{equation}

L'errore su d si ottiene dalla propagazione degli errori associati alla misura degli angoli. Si ha dunque:

\begin{sagesilent}
import numpy as np
import matplotlib.mlab as ml

dati = np.recfromcsv('dati/spettro-reticolo.csv')
var('n')
var('dtheta', latex_name=r'\Delta\theta')
var('lam', latex_name=r'\lambda')
var('sigma_dtheta', latex_name=r'\sigma_{\Delta\theta}')
d(dtheta)=n*lam/(sin(dtheta))
derror = (d.diff())*sigma_dtheta
sigmadth=2*0.0087

dfasterror = fast_float(derror)
#print derror.subs(lam=1,n=2,dtheta=1.).n()

# dati['deltatheta'] = dati['deltatheta']*10^(-3)

def ret_error(dato):
  deltheta = dato['deltatheta']
  res = derror(dtheta=deltheta).subs(lam=dato['lambda']*10^-9, n=int(dato['n']),
                    sigma_dtheta=sigmadth)
  return abs(res[0])
  
errorsarr = np.array([ret_error(dato) for dato in dati ])

dcalc = np.array([float(d.subs(lam=dato['lambda']*10^-9,n=int(dato['n']),dtheta=dato['deltatheta']).n(digits=2) ) for dato in dati ])
print dati
print dcalc
print errorsarr
dati = ml.rec_append_fields(dati, "d", dcalc*10^6)
dati = ml.rec_append_fields(dati, "error_d", errorsarr*10^6)

def stampa_dati(datiarr, header):
  s = r"\begin{tabular}{c*{" + "%d" % (len(datiarr.dtype)-1)
  s += r"}{|c}}"
  s += "%s \\\\" % (header)
  s += r"\midrule"
  for i in range(0, len(datiarr)):
    a = ["%.3G" %x for x in datiarr[i]]
    s += "%s \\\\" % join(a, "&")
  s += r"\end{tabular}"
  return s
\end{sagesilent}
$$d:\sage{d}$$
Derivata e moltiplicata per $\sigma_{\Delta\theta}$, otteniamo $\sigma_d$:
$$\sigma_d: \sage{derror}$$
Per i nostri dati:
\begin{center}
\sagestr{stampa_dati(dati, r'n & $\lambda$ (nm) & $\Delta\theta$ (rad) & d ($\mu$m)& $\sigma_d$ ($\mu$m)' )} 
\end{center}

\begin{sagesilent}
media = np.average(dati['d'], weights=1./dati['error_d']**2)
errmedia = 1./np.sqrt(sum(1./dati['error_d']**2))
\end{sagesilent}

Otteniamo $d=\sage{round(media, 3)}\pm\sage{errmedia.n(digits=2)}$ $\mu$m.




\section*{Prisma}


Nella seconda parte dell'esperienza intendiamo verificare la legge di Cauchy:
\begin{equation}
n = a + \frac{b}{\lambda^2}
\label{Cauchy}
\end{equation}
che lega l'indice di rifrazione di un materiale alla lunghezza d'onda della luce incidente, nel caso dell'indice di rifrazione di un prisma di vetro.
I coefficienti a e b che figurano in \ref{Cauchy} vengono determinati con il metodo dei minimi quadrati. \\
Pertanto misuriamo n corrispondente alle differenti righe spettrali del mercurio (utilizziamo una lampada Hg come sorgente) attraverso la legge:
\begin{equation}
n = \frac{sin(\frac{\alpha + \delta}{2})}{sin(\frac{\alpha}{2})}
\label{n}
\end{equation}
in cui $\alpha$ è l'angolo al vertice del prisma, di $60°$, e $\delta$ l'angolo di deviazione minima. Nota: al denominatore del membro di destra si è trascurato di mettere $n_{aria}=1$.

Con tale procedura otteniamo i seguenti valori per i parametri:


\begin{sagesilent}
 #Fit per ricavare i parametri della relazione di Cauchy
import numpy as np
from scipy import odr
import matplotlib.pyplot as plt

dati = np.recfromcsv('dati/spettro-cauchy.csv')

l = 1./(dati['l']**2)
n = dati['n']

var('x,P')

def func(P,x):
    return P[0]*x+P[1]
    
mymodel = odr.Model(func)

mydata = odr.RealData(l,n)

myodr = odr.ODR(mydata, mymodel, beta0=[1.,1.],  maxit=5000)
myout = myodr.run()

myout.pprint()

def chiquad(xdata, ydata, yfunc, ysigma = (), param=()):
    yteo = yfunc(xdata, *param)
    ddof = len(xdata) - 1 - len(param)
    if (np.isscalar(ysigma)):
        ys = np.ones_like(xdata)*ysigma
    else:
        ys = ysigma
 
    if (len(ys)):
        return sum(((ydata-yteo)/ys)**2)
    else:
        return sum((ydata-yteo)**2/yteo)
        
chiquadrato = chiquad(np.array(l, n, func, ysigma=float(0.05), param=[pars[0]]) 

   
plt.clf()
plt.plot(l,n,'ro')
xin = np.arange(0.9*min(l),1.3*max(l),2*max(l)/10) 
yin = func(myout.beta,xin)
plt.plot(xin,yin,'--')
plt.grid(True,which="both")
plt.savefig("grafici/spettro-cauchy.png", dpi=300)
 

\end{sagesilent}


$$a = \sage{myout.beta[1]} \pm \sage{myout.sd_beta[1]}$$
$$b = \sage{myout.beta[0]} \pm \sage{myout.sd_beta[0]}$$

Dati raccolti:


\begin{sagesilent}
import numpy as np
import matplotlib.mlab as ml

dati = np.recfromcsv('dati/spettro-cauchy.csv')

var('n, alpha')
var('delta', latex_name=r'\delta')
var('sigma_delta', latex_name=r'\sigma_{\delta}')

d(delta)=sin((alpha+delta)/2)/sin(alpha/2)

derror = (d.diff())*sigma_delta
sigmadel=0.0087
alpha_nostro=1.047


def ret_error(dato):
  res = derror(delta=dato['delta']).subs(alpha=alpha_nostro, n=int(dato['n']),
                    sigma_delta=sigmadel)
  return abs(res[0])
  
errorsarr = np.array([ret_error(dato) for dato in dati ])

dcalc = np.array([float(d(delta=dato['delta']).subs(alpha=alpha_nostro, n=int(dato['n'])).n(digits=2) ) for dato in dati ])

print dati
print dcalc
print errorsarr
dati = ml.rec_append_fields(dati, "error_d", errorsarr)

def stampa_dati(datiarr, header):
  s = r"\begin{tabular}{c*{" + "%d" % (len(datiarr.dtype)-1)
  s += r"}{|c}}"
  s += "%s \\\\" % (header)
  s += r"\midrule"
  for i in range(0, len(datiarr)):
    a = ["%.3G" %x for x in datiarr[i]]
    s += "%s \\\\" % join(a, "&")
  s += r"\end{tabular}"
  return s
\end{sagesilent}

\begin{center}
\sagestr{stampa_dati(dati, r"$\lambda$ (nm) & $n$ & $\delta$ & $\sigma_n$")}
\end{center}


In cui l'errore su $n$ è dato dall'errore su $\delta = 0.0087 rad $ tramite la:

$$\sigma_n: \sage{derror}$$



\begin{center}
\includegraphics[scale=0.75]{grafici/spettro-cauchy.png}
\end{center}

\begin{comment}
\begin{sagesilent}
import numpy as np
 
def chiquad(xdata, ydata, yfunc, ysigma = (), param=()):
    yteo = yfunc(xdata, *param)
    ddof = len(xdata) - 1 - len(param)
    if (np.isscalar(ysigma)):
        ys = np.ones_like(xdata)*ysigma
    else:
        ys = ysigma
 
    if (len(ys)):
        return sum(((ydata-yteo)/ys)**2)
    else:
        return sum((ydata-yteo)**2/yteo)
        
chiquadrato = chiquad(np.array(l, n, func, ysigma=float(0.05), param=[pars[0]])    

        
\end{sagesilent}
\end{comment}



\section*{Determinazionedi una sorgente ignota}
Infine utilizziamo i procedimenti e i valori stimati nelle prime parti dell'esperienza  per determinare, mediante lo spettro, il gas corripondente alla sorgente ignota.\\

Calcoliamo n dalla \ref{n}, ed utilizziamo tali valori per ricavare le lunghezze d'onda dalla \ref{Cauchy}, dove adesso sono noti i parametri a e b:
\begin{equation}
\lambda = \sqrt{\frac{b}{n-a}}
\end{equation}


%Elaborazione dati:


\begin{sagesilent}
import numpy as np
import matplotlib.mlab as ml

dati = np.recfromcsv('dati/spettro-neon.csv')

var('alpha')
var('delta', latex_name=r'\delta')
var('sigma_delta', latex_name=r'\sigma_{\delta}')


n(delta) = sin((alpha+delta)/2)/sin(alpha/2)

nerror = (n.diff())*sigma_delta
sigmadel=0.0087
alpha_nostro=1.047
coeff1=9333.5
coeff2=1.5919
sigma_a = 0.000766
sigma_b = 164.4


def ret_error(dato):
  res = nerror(delta=dato['delta']).subs(alpha=alpha_nostro, sigma_delta=sigmadel)
  return abs(res[0])
  
errorsarr = np.array([ret_error(dato) for dato in dati ])

ncalc = np.array([float(n(delta=dato['delta']).subs(alpha=alpha_nostro).n(digits=2) ) for dato in dati ])

#Determinazione delle lambda incongite

var('a', 'b', 'enne')

l(enne)=sqrt(b/(enne-a))

lcalc= np.array([float(l(enne=dato,digits=4).subs(b=coeff1, a=coeff2)  ) for dato in ncalc ])

#Propagazione errori

#var('sigma_a', 'sigma_b')

#lerror(a,b)=sqrt(b/(enne-a))

#dif = lerror.differentiate()

#dif=vector([lerror.diff(a), lerror.diff(b)])

#def errscal(dato):
 # errs = vector([sigma_a, sigma_b])
 # slam = vector([0, 0])
 # diffVal = dif(a=coeff1, b=coeff2, digits=4).subs(enne=dato)
  
 # for i in range(0, len(dif)):
 #     slam[i] = (diffVal[i]*errs[i])
    
 # return abs(slam)

#print 'ncalc', ncalc

#errorl = np.array([float(errscal(trick) ) for trick in ncalc ])



print dati
print ncalc
print errorsarr
print lcalc
#print errorl
dati = ml.rec_append_fields(dati, "n", ncalc)
dati = ml.rec_append_fields(dati, "error_n", errorsarr)
dati = ml.rec_append_fields(dati, "l", lcalc)
#dati = ml.rec_append_fields(dati, "error_l", errorl)

def stampa_dati(datiarr, header):
  s = r"\begin{tabular}{c*{" + "%d" % (len(datiarr.dtype)-1)
  s += r"}{|c}}"
  s += "%s \\\\" % (header)
  s += r"\midrule"
  for i in range(0, len(datiarr)):
    a = ["%.3G" %x for x in datiarr[i]]
    s += "%s \\\\" % join(a, "&")
  s += r"\end{tabular}"
  return s
\end{sagesilent}

\begin{center}
\sagestr{stampa_dati(dati, r"  $\delta$ & $n$ & $\sigma_n$ & $\lambda$  ") }
\end{center}


%\begin{sagesilent}
import matplotlib.pyplot as plt
import numpy as np
import scipy.optimize as opt
\end{sagesilent}


\chapter{Microonde}

\section{Riflessione}

Intendiamo verificare la legge di Cartesio utillizando una lastra di metallo come superficie rifelttente, posizionandola su un supporto magnetico sul goniometro.

\begin{equation}
sin(\theta_{incidente}) = sin(\theta_{rifratto})
\end{equation}

\begin{sagesilent}
dati = np.recfromcsv('dati/MICROONDE/cartesio.csv')
\end{sagesilent}

\begin{center}
\sagestr{stampa_dati(dati, r'$\theta_{inc}$ (rad) & $\theta_{rif}$ (rad) ' )}
\end{center}


\section{Misura della lunghezza d'onda}

\begin{sagesilent}
dati = np.recfromcsv('dati/MICROONDE/lambda.csv')

#Calcolo lambda:

d = np.array(dati['distanza'])
d.astype(float)
n = np.array(dati['enne'])
n.astype(float)
lam = 2*d/n
sigmad = 0.5
errlam = 2/n*sigmad

lmedia = np.average(lam)
confronto = abs(2.85-lmedia)


\end{sagesilent}

Se si pongono trasmettitore e ricevitore ad una distanza di $\frac{n \lambda}{2}$, si può osservare che essi generano onde stazionarie. Infatti le antenne delle due apparecchiature riflettono parzialmente le onde che ricevono, e se li si mette in una condizione tale per cui le onde riflesse hanno la stessa fase delle onde incidenti si osservano dei massimi, viceversa, se le fasi sono antagoniste, si avrà una condizione di nodo.
Ricaviamo la lunghezza d'onda posizionando emettitore e ricevitore ai capi del metro, e muovendo il ricevitore lungo questo osserviamo  il passaggio alterno per massimi e minimi di intensità. Ricaviamo $\lambda$ dalla nota formula per le onde stazionarie:
%\begin{equation}
%d = n\frac{\lambda}{2}
%\label{d}
%\end{equation}
Otteniamo una lambda media: 

$$ \sage{sigmad} \pm \sage{sigmad}$$

%Dal confronto con la $\lambda$ (2.85 cm) teorica otteniamo: $\lambda_{teo} - \lambda_{cal} || = \sage{confronto} < 2 \sigma ( = 2 \sage{errlam} $


\section{Rifrazione attraverso un prisma}

In questa parte dell'esperienza si verifica la legge di snell:
\begin{equation}
\frac{sin(\theta_{i})}{sin(\theta_{r})} = \frac{n_2}{n_1}
\end{equation}
A tal fine posizioniamo una pedana di polistirolo sul goniometro, e vi poniamo sopra un prisma, sempre di polistirolo, contenente "styrene pellets". Si verifica la legge per vari massimi di intensità (Nota: $n_1$ = 1).

%Dati???

\section{Polarizzazione}

\begin{sagesilent}
#Calcolo dell'intensità con la legge di Malus

dati = np.recfromcsv('dati/MICROONDE/malus.csv')

izero = 8.75
gamma=dati['gamma']*3.14/180
iteorico = izero*(cos(gamma))**2

dati = ml.rec_append_fields(dati, 'iteo', iteorico)

\end{sagesilent}


Le microonde uscenti dal trasmettitore sono polarizzate linearmente, per cui soddisfano alla legge di Malus che lega l'intensità all'angolo tra la normale al ricevitore e la direzione dell'onda. Fissata una distanza, posizioniamo emettitore e ricevitore uno difronte all'altro e misuriamo la diminuzione d'intensità relativa al variare dell'angolo rispetto all'orizzontale di emettitore-ricevitore. Di seguito i dati letti sul multimetro a confronto con l'intensità teorica data dalla legge di Malus:

\begin{equation}
I = I_{0} cos^2 \gamma
\end{equation}

\begin{center}
%\sagestr{stampa_dati(dati, r'$I$ (Volt) & $\gamma$ (rad) & $I_{teorico}$ (Volt) ' )}
\end{center}


\section{Interferenza da doppia fenditura}
\begin{sagesilent}
dfen = np.recfromcsv("dati/microonde-doppiafen.csv")
dfen.sort()
var('x,a,w,t')
model(x, a, w, t) = a*sin(w*x+t)

def ff(x, a,w,t):
  f = fast_callable(model)
  return f(x,a,w,t)

#print dfen['angolo']
pars, pcov = opt.curve_fit(ff, np.array(dfen['angolo']), np.array(dfen['volt']), p0=[500., 50., 1.])

print pcov

print pars
plt.clf()
xin = np.arange(min(dfen['angolo']), 220, 1)
print "now"
yin = ff(xin, pars[0], pars[1], pars[2])
plt.plot(xin, yin)
plt.plot(dfen['angolo'], dfen['volt'], 'o--')

plt.savefig("grafici/microonde-doppiafend.png")
\end{sagesilent}


Utilizzando un supporto magnetico sul goniometro, montiamo tre lastre di metallo in modo da avere due
fenditure di circa 1,5 cm. Spostando il ricevitore si troviamo i massimi e i minimi di interferenza al fine di verificare la
loro posizione rispetto alla previsione teorica: $d \sin(\theta) = n \lambda$ (per i massimi, con d distanza delle fenditure, $\theta$ angolo
rispetto alla perpendicolare, n numero intero).

\includegraphics[scale=0.75]{grafici/microonde-doppiafend.png}

\section{Specchio di Lloyd}

\begin{sagesilent}
#Calcolo della lunghezza d'onda

dati = np.recfromcsv('dati/MICROONDE/lloyd.csv')

d = dati['delta']
enne = dati['n']

lam = 2*d/enne


dati = ml.rec_append_fields(dati, 'lung_onda', lam)

\end{sagesilent}

Posizioniamo emettitore e ricevitore a distanza di un circa un metro, uno dinnanzi all'altro. Su un secondo metro, perpendicolare alla direzione emettitore-ricevitore e passante per il goniometro, montiamo una lastra di metallo che funge da specchio: in tal modo generiamo un'interferenza dovuta alla differenza di cammino delle onde (una parte arriva direttamente al ricevitore, mentre altre onde percorrono il cammino dall'emettitore allo specchio e dallo specchio al ricevitore). Cerchiamo il minimo di intensità con lo specchio a distanza minima dal goniometro e raccogliamo i massimi di intensità al variare della distanza, per determinare $\lambda$ come in \ref{d}

\begin{center}
%\sagestr{stampa_dati(dati, r'$d$ (cm) & $n$ & $\lambda$ ' )}
\end{center}

\section{Diffrazione di Bragg}

\begin{sagesilent}
#Grafico dell'intensità in funzione dei gradi

dati = np.recfromcsv('dati/MICROONDE/bragg.csv')
plt.clf()
plt.plot(dati['theta'], dati['volt'],'ro')

plt.savefig("grafici/microonde-bragg.png")

\end{sagesilent}


Nell'ultima parte dell'esperimento ci proponiamo di verificare la legge di Bragg, 
\begin{equation}
n \lambda = 2 d \sin(\theta)
\end{equation}
la quale esprime analiticamente i fenomeni di interferenza causati dalla riflessione di onde su piani paralleli di un reticolo cristallino. Per riprodurre il fenomeno, facciamo incidere le microonde su un cubo di polistirolo su cui sono inserite su spaziature regolari delle sferette di acciaio.

Nella formula:\\
$\theta$ è l'angolo che il fascio incidente forma col piano cristallino\\
$\lambda$ è la lunghezza d'onda della radiazione\\
$d$ è la distanza tra due piani adiacenti\\
$n$ indica l'ordine della diffrazione (tipicamente solo quello per n=1 è apprezzabile).\\

La formula si spiega in maniera analitica considerando una differenza di cammino ottico pari a $2d\sin(\theta)$.


%\includegraphics[scale=0.75]{grafici/microonde-bragg.png}

\section{Analisi dati}

\section{Allegato: dati}
\begin{sagesilent}
def stampa_dati(wa, header):
  s = r"\begin{tabular}{c*{" + "%d" % (len(wa.dtype)-1)
  s += r"}{|c}}"
  s += "%s \\\\" % (header)
  s += r"\midrule"
  for i in range(0, len(wa)):
    a = ["%s" %x for x in d[i]]
    s += "%s \\\\" % join(a, "&")
  s += r"\end{tabular}"
  return s
\end{sagesilent}

\begin{center}

% \sagestr{stampa_dati(d, r"Frequenza (Hz) & I (mA) & Valore mu & $V_{rms}$ (V)")}
\end{center}


%

\chapter{Dati geometrici}
\begin{center}
\begin{tabular}{|c|c|c|c|c|c|c|c|}
\midrule
\textit{Solenoide} & $n$ & $R_{int}$ & $R_{ext}$ & $R_{medio}$ & $l_{circonferenza}$ & $h$ & $A_{sezione}$  \\
		   & 322 & 0,05	 & 0,0937  & 0,07185 & 0,451446864 & 0,0225 &	0,00098325 \\
 \midrule
\end{tabular}
\end{center}


\begin{center}
\begin{tabular}{|c|c|c|c|c|}
\midrule
\textit{Capacità} & $R_{int}$ (cm) & $R_{ext}$ (cm) & $lunghezza$ (cm) & $spessore piastre$ (cm)\\
   & 5.89 & 3.546  & 40  & 0.31 \\
%?? nel file ods ci sono altre misure per gli stessi campi!
 \midrule
\end{tabular}
\end{center}

\subsubsection{Misura della lunghezza d'onda del laser}

Misuriamo c dalla relazione:
\begin{equation}
 c=\frac{1}{sqrt{LC}}
\end{equation}
pertanto misuriamo la frequenza di risonanza di un cicuito RLC, per differenti configurazioni del circuito.

\subparagraph*{C ed L}
Nella prima parte dell'esperienza abbiamo verificato che i valori di capacità ed induttanza calcolati a partire dai parametri geometrici corrispondessero effettivamente a quelli misurati in laboratorio. 
Per tanto, in entrambi i circuiti, calcoliamo algebricamente i valori di V e $V_[0]$ per cui $t = \tau$. Inseriamo tali valori e misuriamo i tempo di scarica. Dalla stima del tempo caratteristico ricaviamo L e C. La capacità risulta in accordo, ed è stimata intorno ai $4.38$ nF, l'induttanza invece risulta:

\begin{center}
\begin{tabular}{c c}
$L_{geometrico}$  & 0.000293\\
$L_{misurato}$ & 0.000325\\
\end{tabular}
\end{center}

Visto che è presente una discrepanza tra i valori misurati e quelli geometrici, dovuto a capacità parassite, d'ora in avanti lavoreremo con una induttanza equivalente (modificando un parametro geometrico per ottenere l'accordo): $N_{eq} = 339$


\begin{comment}
Costruiamo un circuito RLC. Nota: il condensatore risente molto delle fluttuazioni, per cui lo colleghiamo a terra, per ridurle il più possibile.

Frequenze di risonanza intorno ai 500 kH (nota: è un minimo -> potenziale ai capi di LC)
\end{comment}

\subparagraph*{L=40 cm}
\begin{sagesilent}
#Fit per stimare i parametri -> Frequenza di risonanza: -b/2a

dati = np.recfromcsv('dati/em-parabola1.csv')


\end{sagesilent}



\begin{comment}
\begin{center}
\sagestr{stampa_dati(dati, r'$\omega$ (rad) & $\V_{out}$ (nm) & $\sigma_{V_{out}}$ (nm)' )}
\end{center}
\end{comment}

\subparagraph*{L=38 cm}

\subparagraph*{L=34 cm}

\subparagraph*{L=32 cm}

\subparagraph*{L=28 cm}

\chapter{Analisi con capacità parassite in serie}
\chapter{Analisi con capacità parassite in parallelo}



%
\chapter{Interferometro (O2)}

\subsubsection{ strumenti}


\subsubsection{Misura della lunghezza d'onda del laser}

$$ \lambda = \frac{2d}{m} $$

con $d$ spostamento dello specchio mobile e $m$ numero di massimo attraversati (visualizzati sullo schermo).

\begin{center}

\begin{tabular}{c c}
\textbf{Fabry-Perot} & \hspace{2cm} \textbf{Michelson}\\
\\
\begin{tabular}{c|c|c|c|c|c}
d ($\mu m$)& $\sigma_d$ & m & $\sigma_m$ & $\lambda$ ($nm$) & $\sigma_{\lambda}$\\
\midrule
29 & 1 & 87 & 4 & 670 & 38\\
25 & 1 & 79 & 1 & 632 & 26\\
31 & 1 & 81 & 1 & 642 & 26\\
26 & 1 & 75 & 7 & 693 & 70\\
20 & 1 & 64 & 5 & 625 & 58\\
35 & 1 & 107 & 5 & 654 & 36\\


?? \\

\end{tabular}

& \hspace{2cm}

\begin{tabular}{c|c|c|c|c}
d ($\mu m$)& m & $\sigma_m$ & $\lambda$ ($nm$) & $\sigma_{\lambda}$\\
\midrule
25 & 75 & ? & 666 & ?\\
75 & 72 & ? & 697 & ?\\

\end{tabular}

\end{tabular}

\end{center} 

In cui $\sigma_{\lambda}$ è data dalla propagazione dell'errore su d e m:

$$ \sigma_{\lambda} = \sqrt{ ( \frac{\partial f}{\partial d} \sigma_{d} )^2 + ( \frac{\partial f}{\partial m} \sigma_{m} )^2 } $$

Valore atteso: 632.8 $nm$\\

Dalla media pesata otteniamo una $\lambda$ di:

$$ \lambda =\displaystyle \sum{\frac{\frac{x_i}{(\sigma_i)^2}}{\frac{1}{(\sigma_i)^2}}} \pm \displaystyle\sum{\frac{1}{(\sigma_i)^2}} = 644 \pm 14 nm $$



\subsection{Misura dell'indice di rifrazione dell'aria}

Si pone una cella a vuoto (spessore $d=3 cm$) tra lo specchio semi-riflettente e lo specchio rotante (che in questa parte dell'esperienza rimane fisso) nell'interferometro di Michelson. La cella è collegata ad un compressore, e un barometro segna la variazione di pressione.
Raccogliamo il numero di massimi che attraversano lo schermo al variare della pressione e ricaviamo $n_{aria}$ dalla relazione:
$$ n = mP+1 $$
$$ m = \frac{n_i - n_f}{P_i-P_f} = \frac{N \lambda_0}{2d(P_i-P_f}$$

\begin{center}
\begin{tabular}{c|c|c}
$P_i-P_f$ ($kPa$) & N & m \\
\midrule
70 & 19 & $2.859\cdot10^{-6}$\\
\end{tabular}
\end{center}

$$ n = mP+1 = 1.000289 \pm $$

Valore atteso: 1.000293

\subsection{Misura dell'indice di rifrazione del vetro}

In modo analogo a quanto fatto per la misura di $n_{aria}$, si pone tra lo specchio semi-riflettente e lo specchio rotante una lastra di vetro (spessore $t= 5 mm$). Raccogliamo il numero N di frange di interferenza contate per uno spostamento angolare superiore a $10°$, e ricaviamo $n_{vetro}$ dalla relazione:

$$n=\frac{(2t-N\lambda_0)(1-cos(\theta))}{2t(1-cos\theta)-N\lambda_0}$$

Per fissare lo zero, troviamo l'angolo di deviazione minima.

$\delta = 0.6 \pm (?) $

\begin{center}
\begin{tabular}{c|c|c|c|c|c}
$\theta (°) $ & $\sigma_{\theta}$ & N & $\sigma_{N}$ & n & $\sigma_{n}$\\
\midrule
10.8 & 3 & 112 & 3 & 1.578 & ? \\
9.2 & 3 & 87 & 2 & 1.529 & ? \\
\end{tabular}
\end{center}

\subsection{Reticolo ad incidenza radente}

\begin{sagesilent}

import numpy as np
import matplotlib.mlab as ml

dati = np.recfromcsv("dati/interferometro.csv")

var('d, h, l, k,n, sigma_h, sigma_l,sigma_k')
lamb(h,k,l) = (d/n)*(cos(atan(h/l))-cos(atan(k/l)))


elle = 344

dif = lamb.differentiate()
errs = vector([sigma_h, sigma_l, sigma_k])
slam = vector([d, h, l])

for i in range(0, len(dif)):
    slam[i] = (dif[i]*errs[i])
    
errscal = abs(slam)
\end{sagesilent}


In questa parte dell'esperienza misuriamo la $\lambda$ della luce a laser attraverso considerazioni geometriche riguardo ai fenomeni di interferenza e riflessione. Infatti, noto il passo del reticolo (nel nostro caso il reticolo consiste in un righello metallico, e dunque il passo è $d = 1 mm$), possiamo ricavare $\lambda$ dalla relazione:

$$ n\lambda = d(cos\theta_i-cos\theta_r) $$

con $l = 344 cm $ (distanza sorgente-schermo) e $\theta = arctg\frac{h}{l}$

Per considerare l'errore che commettiamo su $l$, dunque, la funzione che ci interessa è:
$$\lambda = \sage{lamb}$$
Il cui differenziale è:
$$\nabla \lambda = \sage{dif}$$

Dunque, l'incertezza su $\lambda$ è data dalla relazione
$$\sigma_\lambda = \sage{errscal}$$


\begin{sagesilent}
subd = errscal.subs(d=0.1,h=30,l=elle,sigma_h=0.3,sigma_k=0.3,sigma_l=1)
sublambda = lamb.subs(d=0.1,h=30,l=elle)

lambdas = [abs((sublambda.subs(h=int(dato['h']), k=dato['k'], n=int(dato['n']))*10^7)).n(digits=4) for dato in dati]
errorsarr = np.array([(subd.subs(n=int(dato['n']), k=dato['k'])*10^7).n(digits=4) for dato in dati], dtype='f4')
print errorsarr

dati = ml.rec_append_fields(dati, "lambda", lambdas)
dati = ml.rec_append_fields(dati, "sigma_lambda", errorsarr)
print dati

def stampa_dati(wa, header):
  s = r"\begin{tabular}{c*{" + "%d" % (len(wa.dtype)+1)
  s += r"}{|c}}"
  s += "%s \\\\" % (header)
  s += r"\midrule"
  for i in range(0, len(wa)):
    a = ["%.4G" % x for x in wa[i]]
    s += "%s \\\\" % join(a, "&")
  s += r"\end{tabular}"
  return s
\end{sagesilent}

Sostituendo:
\begin{center}
\sagestr{stampa_dati(dati, "n & k &h & $\lambda$ & $\sigma_\lambda$ ")}
\end{center}
Graficati:

%\chapter{Misura della velocità della luce}

L'obiettivo del nostro esperimento è misurare la velocità della luce $c$.

L'apparato di misurazione consiste principalmente in:
\begin{itemize}
 \item Un laser Elio-Neon ($\lambda=32$ nm).
 \item Uno specchio rotante a velocità angolare regolabile.
 \item Due lenti convergenti, di lunghezza focale $l_1 = 48mm$ e $l_2 = 252 mm$.
 \item Un microscopio con beam splitter e micrometro.
\end{itemize}

Lo specchio rotante viene fatto girare con velocità angolare $\omega_1$ in senso orario e $\omega_2$ in senso antiorario. Queste velocità angolari sono identiche nella maggior parte dei casi, e la loro differenza è trascurabile.

Chiamando $s_{cw}$ e $s_{ccw}$ rispettivamente le misure con specchio rotante in senso orario e antiorario, per come è orientata la strumentazione il numero
$$\Delta s = s_{cw} - s_{ccw}$$
deve essere positivo.

Purtroppo però non è questo il caso per qualche misura presa in mattinata, per ragioni sconosciute e che non siamo riusciti a riprodurre. Questi dati sono esclusi dalla misurazione in quanto evidenti errori, e sono mostrati in rosso nel grafico seguente
I dati sono molti, dunque forniamo qui soltanto una visione grafica, e lasciamo la tabella come allegato.

\includegraphics[scale=0.75]{grafici/C/dati.png}

\section{Analisi dati}

Data una 
m=0.04594423946498933

Ricaviamo, tramite  il fit della funzione $V=R*I$" dove R è parametro da stimare, due resistenze ignote.


\includegraphics[scale=0.75]{grafici/C1/res1.png}
\

\includegraphics[scale=0.75]{grafici/C1/res2.png}

Misuro la resistenza interna del voltometro, mantenendo costante la ddp a $14.5\ V$ e variando la resistenza all'interno del circuito. 
Il voltmetro è in parallelo al circuito, perciò $R_i$:

$$R_i = \frac{RV}{RI-V} $$

dove R è la resistenza variabile, I la corrente nel circuito e $V= 14.5\ V$

\begin{center}
\begin{tabular}{*{2}{c}}
\end{tabular}

La resistenza risulta $9.20 \pm 0.27 \ M \Omega$.


Misuriamo la resistenza interna dell'amperometro, che è collegato in serie al circuito. 

$$R_i = \frac{V-RI}{I}$$

In questo caso, R è fissato ($R=0.5 \Omega$) e sono V e I a variare

La resistenza interna risulta $11.42 \pm 0.45 \Omega$


\end{center}


Colleghiamo una piccola lampada a filamento al circuito, e verifichiamo che il suo comportamento resistivo non segue la legge di Ohm. 








\end{document} 



