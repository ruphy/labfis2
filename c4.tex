\chapter{C4}
\section{Prima parte}

In regime di corrente impulsata, cerchiamo di determinare i parametri $\gamma$ e $\omega_0$ del circuito.

\begin{sagesilent}

import numpy as np
from scipy import odr
import matplotlib.pyplot as plt


\end{sagesilent}
\subsection{Sottosmorzamento}

\begin{sagesilent}
#Sottosmorzamento tempo-volt

t =np.array([190,570,950,1520])
picchi=np.array([520,-148,60,-24])

def func(P,x):
    print P
    return 100*np.sin(x)
    ial = P[0]*P[1]*P[2]*np.exp(-1*P[3]*x)*(P[3]*np.cos(P[4]*x)+P[4]*np.sin(P[4]*x))
    print max(ial)
    return ial
    #return f(x, P[0], P[1], P[2], P[3], P[4])

var('x,r,c,v,g, om')
def f(x,r,c, v, g, om):
    return r*c*v*exp(-g*x)*(g*cos(om*x)+om*sin(om*x))
    

richipars = [419., 9.554*10^-8, 27.93, 5896., 3.173*10^4]
xin = np.arange(150,2000,0.1)

yin = func(richipars, xin)

plt.clf()
plt.plot(t,picchi,'bo')
plt.plot(xin,yin,'g--')
plt.ylabel(r"$V(t)$ [volt]")
plt.xlabel(r"Tempo [s]")
plt.savefig("grafici/C4-om.png",dpi=300)
 
\end{sagesilent}

Nel sotto, i parametri trovati dal fit sono stati:
$$V_0 = 27.93\pm3.837$$
$$R = 419 \pm 57.56$$
$$C = 9.554\times10^{-8} \pm 1.3\times10^{-8}$$
$$\omega_0 = 3.173\times10^4 \pm 84.57$$
$$\gamma = 5896 \pm 471.8$$

Dalla funzione:
$$V(t) = RCV_0e^{-\gamma t}[\gamma cos(\omega t) + \omega sin(\omega t)]$$
ove
$$\omega=\sqrt{\omega_0^2-\gamma^2}$$
In un grafico:


\begin{center}
 \includegraphics[scale=0.75]{grafici/C4-om.png}
\end{center}
\subsection{Smorzamento critico}

\begin{sagesilent}

#Smorzamento critico tempo-volt
dati = np.recfromcsv("dati/C4/C4-smorzamento.csv")
t = dati['t']
v = dati['v']

def sovrasm(p, x):
  gamma = p[2]
  deway = p[0]*p[1]*(gamma**2)*x*np.exp(-1*(gamma)*x)
  return deway
  
#(myout, chiquad, prob) = fit_chiquad(t, v, sottosm, [0.02, 500,0.031,0.29], v*0.04)
#myout.pprint()

richipars = [0.001193, 8.62*10^5, 0.033]

plt.clf()
xin = np.arange(2, 200)
yin = sovrasm(richipars, xin)
plt.plot(xin, yin, '--')
plt.errorbar(t, v, v*0.04, fmt="g.")
plt.xlabel('Tempo (s)')
plt.ylabel('D.d.p (volt)')
plt.savefig("grafici/C4-sm-critico.png", dpi=300)
\end{sagesilent}


\begin{center}
\includegraphics[scale=0.75]{grafici/C4-sm-critico.png}
\end{center}

\subsection{Sovrasmorzamento}
\begin{sagesilent}
#Sovrasmorzamento tempo-volt
dati = np.recfromcsv("dati/C4/C4-sovrasmorzamento.csv")
t = dati['t']
v = dati['v']

def sottosm(p, x):
  omega0 = p[2]
  gamma = p[3]
  beta = np.sqrt(abs(gamma**2-omega0**2))
  deway = p[0]*p[1]*(omega0**2)*(np.exp(-1*(gamma-beta)*x) - np.exp(-1*(beta+gamma)*x))/(2*beta)
  return deway
  
#(myout, chiquad, prob) = fit_chiquad(t, v, sottosm, [0.02, 500,0.031,0.29], v*0.04)
#myout.pprint()

richipars = [0.0266, 4.581*10^5, 0.031, 0.2917]

plt.clf()
xin = np.arange(2, 700)
yin = sottosm(richipars, xin)
plt.plot(xin, yin, '--')
plt.errorbar(t, v, v*0.04, fmt="g.")
plt.xlabel('Tempo (s)')
plt.ylabel('D.d.p (volt)')
plt.savefig("grafici/C4-sovrasmorzamento.png", dpi=300)
\end{sagesilent}

Per quanto riguarda il sovrasmorzamento, i parametri trovati dal fit sono stati:
$$Q_0 = 0.0264\pm0.0003815$$
$$R = 4.581\times10^5 \pm 6559$$
$$\omega_0 = 0.03103 \pm 0.0009811$$
$$\gamma = 0.2917 \pm 0.01839$$

Dalla funzione:
$$V(t) = Q_0R\frac{\omega_0}{2\beta}[e^{-(\gamma-\beta)t}-e^{-(\gamma+\beta)t}]$$
ove
$$\beta=\sqrt{\gamma^2-\omega_0^2}$$
In un grafico:

\begin{center}
\includegraphics[scale=0.75]{grafici/C4-sovrasmorzamento.png}
\end{center}



\section{Seconda parte}


\begin{sagesilent}

#Funzione di trasferimento del modulo: freq-volt
#RLC in corrente alternata
dati = np.recfromcsv("dati/C4-tensione-fase.csv")
var('x,l,c,v,w')
r = 300
def f(x, l, w, v):
    return (v*2*3.14*x*w)/( sqrt( (2*3.14*x*w)^2+(w*l/r)^2*((2*3.14*x)^2-w^2)^2 ) )
    
puls = dati['frequenza_hz'] #*2*3.14
ennupla = list(((puls[i]), dati['v_out__v_in'][i]) for i in range(0,len(dati['frequenza_hz'])))


fit = find_fit(ennupla, f, parameters=[l,w,v], variables=[x], solution_dict=True)

print fit

plt.clf()
xin = np.arange(100, 100000, 10)
yin = f(xin, fit[l], fit[w], fit[v])
plt.semilogx(xin,yin, 'g--')
plt.errorbar(dati['frequenza_hz'], dati['v_out__v_in'], yerr=dati['v_out__v_in']*0.02, fmt='.r')
#plt.semilogx(dati['frequenza_hz'], dati['v_out__v_in'], '|b')

plt.xlabel(r"$\omega$ [$rad/s$]")
plt.ylabel(r"$V(\omega)$ [$rad$]")
plt.grid(True)

plt.savefig("grafici/C4-ris.png", dpi=300)

#Funzione di trasferimento fase:freq-rad
  
\end{sagesilent}

\begin{center}
\includegraphics[scale=0.75]{grafici/C4-ris.png}
\end{center}



%Funzione di trasferimento del modulo
% \begin{sagesilent}
% 
% freq = trasfm['frequenza_hz']
% v = trasfm['v_out_v_in']
% 
% #Funzione di trasferimento del modulo
% 
% def fu(P,x):
%     return (P[0]*2*3.14*x*P[1])/( sqrt( (2*3.14*x*P[1])^2+(w*P[2]/R)^2*((2*3.14*x)^2-[1]^2)^2 ) )
% 
%   
% myodr, chi = fit_chiquad(freq, v, fu, ysigma=yerr, param=[mytransm.beta])
% print "Chi quadro %.4f" % chi
%   
% \end{sagesilent}

