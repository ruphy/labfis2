\documentclass[a4paper,10pt]{report}
 

\usepackage{amsmath}
\usepackage{amsthm}
\usepackage{amssymb}
\usepackage{booktabs}
\usepackage[table]{xcolor}
\usepackage{amssymb}
\usepackage[utf8x]{inputenc}

\begin{document}

\chapter{Interferometro (O2)}

\subsubsection{ strumenti}


\subsubsection{Misura della lunghezza d'onda del laser}

$$ \lambda = \frac{2d}{m} $$

con $d$ spostamento dello specchio mobile e $m$ numero di massimo attraversati (visualizzati sullo schermo).

\begin{center}

\begin{tabular}{c c}
\textbf{Fabry-Perot} & \hspace{2cm} \textbf{Michelson}\\
\\
\begin{tabular}{c|c|c|c|c|c}
d ($\mu m$)& $\sigma_d$ & m & $\sigma_m$ & $\lambda$ ($nm$) & $\sigma_{\lambda}$\\
\midrule
29 & 1 & 87 & 4 & 670 & 38\\
25 & 1 & 79 & 1 & 632 & 26\\
31 & 1 & 81 & 1 & 642 & 26\\
26 & 1 & 75 & 7 & 693 & 70\\
20 & 1 & 64 & 5 & 625 & 58\\
35 & 1 & 107 & 5 & 654 & 36\\


?? \\

\end{tabular}

& \hspace{2cm}

\begin{tabular}{c|c|c|c|c}
d ($\mu m$)& m & $\sigma_m$ & $\lambda$ ($nm$) & $\sigma_{\lambda}$\\
\midrule
25 & 75 & ? & 666 & ?\\
75 & 72 & ? & 697 & ?\\

\end{tabular}

\end{tabular}

\end{center} 

In cui $\sigma_{\lambda}$ è data dalla propagazione dell'errore su d e m:

$$ \sigma_{\lambda} = \sqrt{ ( \frac{\partial f}{\partial d} \sigma_{d} )^2 + ( \frac{\partial f}{\partial m} \sigma_{m} )^2 } $$

Valore atteso: 632.8 $nm$\\

Dalla media pesata otteniamo una $\lambda$ di:

$$ \lambda =\displaystyle \sum{\frac{\frac{x_i}{(\sigma_i)^2}}{\frac{1}{(\sigma_i)^2}}} \pm \displaystyle\sum{\frac{1}{(\sigma_i)^2}} = 644 \pm 14 nm $$



\subsection{Misura dell'indice di rifrazione dell'aria}

Si pone una cella a vuoto (spessore $d=3 cm$) tra lo specchio semi-riflettente e lo specchio rotante (che in questa parte dell'esperienza rimane fisso) nell'interferometro di Michelson. La cella è collegata ad un compressore, e un barometro segna la variazione di pressione.
Raccogliamo il numero di massimi che attraversano lo schermo al variare della pressione e ricaviamo $n_{aria}$ dalla relazione:
$$ n = mP+1 $$
$$ m = \frac{n_i - n_f}{P_i-P_f} = \frac{N \lambda_0}{2d(P_i-P_f}$$

\begin{center}
\begin{tabular}{c|c|c}
$P_i-P_f$ ($kPa$) & N & m \\
\midrule
70 & 19 & $2.859\cdot10^{-6}$\\
\end{tabular}
\end{center}

$$ n = mP+1 = 1.000289 \pm $$

Valore atteso: 1.000293

\subsection{Misura dell'indice di rifrazione del vetro}

In modo analogo a quanto fatto per la misura di $n_{aria}$, si pone tra lo specchio semi-riflettente e lo specchio rotante una lastra di vetro (spessore $t= 5 mm$). Raccogliamo il numero N di frange di interferenza contate per uno spostamento angolare superiore a $10°$, e ricaviamo $n_{vetro}$ dalla relazione:

$$n=\frac{(2t-N\lambda_0)(1-cos(\theta))}{2t(1-cos\theta)-N\lambda_0}$$

Per fissare lo zero, troviamo l'angolo di deviazione minima.

$\delta = 0.6 \pm (?) $

\begin{center}
\begin{tabular}{c|c|c|c|c|c}
$\theta (°) $ & $\sigma_{\theta}$ & N & $\sigma_{N}$ & n & $\sigma_{n}$\\
\midrule
10.8 & 3 & 112 & 3 & 1.578 & ? \\
9.2 & 3 & 87 & 2 & 1.529 & ? \\
\end{tabular}
\end{center}

\subsection{Reticolo ad incidenza radente}

In questa parte dell'esperienza misuriamo la $\lambda$ della luce a laser attraverso considerazioni geometriche riguardo ai fenomeni di interferenza e riflessione. Infatti, noto il passo del reticolo (nel nostro caso il reticolo consiste in un righello metallico, e dunque il passo è $d = 1 mm$), possiamo ricavare $\lambda$ dalla relazione:

$$ n\lambda = d(cos\theta_i-cos\theta_r) $$

con $l = 344 cm $ (distanza sorgente-schermo) e $\theta = arctg\frac{h}{l}$

\begin{center}
\begin{tabular}{c|c|c|c|c|c|c|c}

n & h $(cm)$ & $\sigma_h$ & $\Delta h (cm)$ & $\theta (°)$ & $\sigma_{\theta}$ & $\lambda$ & $\sigma_{\lambda}$\\
\midrule

6	& 6.5	& 0.2 &	23.4 & &	1.083043981  & &\\
5	& 12.5	& &	17.2 & &	2.0821086591 & &\\
4	& 16.5	& &	13.1 & &	2.7474869399 & &\\
3	& 23	& &	7    & & 	3.8270695544 & &\\
2	& 25.5	& &	4.6  & &	4.2416126702 & &\\
1	& 27.5	& &	2.3	 & &	4.5729295588 & &\\
0	& 30	& &	0 	 & &	4.9866429981 & &\\
1	& 32	& &	2 	 & &	5.3172412847 & &\\
2	& 34	& &	3.8	 & & 	5.6474852401 & &\\
3	& 37.5	& &	7.5	 & &	6.2244955262 & &\\

\end{tabular}
\end{center}
				






\end{document} 


