
\chapter{Interferometro (O2)}

\subsubsection{ strumenti}


\subsubsection{Misura della lunghezza d'onda del laser}

$$ \lambda = \frac{2d}{m} $$

con $d$ spostamento dello specchio mobile e $m$ numero di massimo attraversati (visualizzati sullo schermo).

\begin{center}

\begin{tabular}{c c}
\textbf{Fabry-Perot} & \hspace{2cm} \textbf{Michelson}\\
\\
\begin{tabular}{c|c|c|c|c|c}
d ($\mu m$)& $\sigma_d$ & m & $\sigma_m$ & $\lambda$ ($nm$) & $\sigma_{\lambda}$\\
\midrule
29 & 1 & 87 & 4 & 670 & 38\\
25 & 1 & 79 & 1 & 632 & 26\\
31 & 1 & 81 & 1 & 642 & 26\\
26 & 1 & 75 & 7 & 693 & 70\\
20 & 1 & 64 & 5 & 625 & 58\\
35 & 1 & 107 & 5 & 654 & 36\\


?? \\

\end{tabular}

& \hspace{2cm}

\begin{tabular}{c|c|c|c|c|c}
d ($\mu m$) & $\sigma_d$ & m & $\sigma_m$ & $\lambda$ ($nm$) & $\sigma_{\lambda}$\\
\midrule
25 & 1& 75 & 2 & 666 & ?\\
75 & 1& 72 & 1 & 697 & ?\\

\end{tabular}

\end{tabular}

\end{center} 

In cui $\sigma_{\lambda}$ è data dalla propagazione dell'errore su d e m:

$$ \sigma_{\lambda} = \sqrt{ ( \frac{\partial \lambda}{\partial d} \sigma_{d} )^2 + ( \frac{\partial \lambda}{\partial m} \sigma_{m} )^2 } $$

Valore atteso: 632.8 $nm$\\

Dalla media pesata con errore otteniamo una $\lambda$ media di:

$$ \lambda =\displaystyle \sum_i{\frac{\frac{x_i}{(\sigma_i)^2}}{\frac{1}{(\sigma_i)^2}}} \pm \displaystyle\sum_i{\frac{1}{(\sigma_i)^2}} = 644 \pm 14 nm $$

\subsection{Misura dell'indice di rifrazione dell'aria}

Si pone una cella a vuoto (spessore $d=3 cm$) tra lo specchio semi-riflettente e lo specchio rotante (che in questa parte dell'esperienza rimane fisso) nell'interferometro di Michelson. La cella è collegata ad un compressore, e un barometro segna la variazione di pressione.
Raccogliamo il numero di massimi che attraversano lo schermo al variare della pressione e ricaviamo $n_{aria}$ dalla relazione:
$$ n = mP+1 $$
$$ m = \frac{n_i - n_f}{P_i-P_f} = \frac{N \lambda_0}{2d(P_i-P_f)}$$

\begin{center}
\begin{tabular}{c|c|c}
$P_i-P_f$ ($kPa$) & N & m \\
\midrule
70 & 19 & $2.859\cdot10^{-6}$\\
\end{tabular}
\end{center}

$$ n = mP+1 = 1.000289 \pm \text{booh, di quanto sono le incertezze?}$$

Valore atteso: 1.000293

\subsection{Misura dell'indice di rifrazione del vetro}

In modo analogo a quanto fatto per la misura di $n_{aria}$, si pone tra lo specchio semi-riflettente e lo specchio rotante una lastra di vetro (spessore $t= 5 mm$). Raccogliamo il numero N di frange di interferenza contate per uno spostamento angolare superiore a $10°$, e ricaviamo $n_{vetro}$ dalla relazione:

$$n=\frac{(2t-N\lambda_0)(1-cos(\theta))}{2t(1-cos\theta)-N\lambda_0}$$

Per fissare lo zero, troviamo l'angolo di deviazione minima.

$\delta = 0.6 \pm (?) $

\begin{center}
\begin{tabular}{c|c|c|c|c|c}
$\theta (°) $ & $\sigma_{\theta}$ & N & $\sigma_{N}$ & n & $\sigma_{n}$\\
\midrule
10.8 & 3 & 112 & 3 & 1.578 & ? \\
9.2 & 3 & 87 & 2 & 1.529 & ? \\
\end{tabular}
\end{center}

\subsection{Reticolo ad incidenza radente}

\begin{sagesilent}

import numpy as np
import matplotlib.mlab as ml

dati = np.recfromcsv("dati/interferometro.csv")

var('d, h, l, k,n, sigma_h, sigma_l,sigma_k')
lamb(h,k,l) = (d/n)*(cos(atan(h/l))-cos(atan(k/l)))


elle = 344

dif = lamb.differentiate()
errs = vector([sigma_h, sigma_k, sigma_l])
slam = vector([d, h, l])

for i in range(0, len(dif)):
    slam[i] = (dif[i]*errs[i])
    
errscal = abs(slam)
\end{sagesilent}


In questa parte dell'esperienza misuriamo la $\lambda$ della luce a laser attraverso considerazioni geometriche riguardo ai fenomeni di interferenza e riflessione. Infatti, noto il passo del reticolo (nel nostro caso il reticolo consiste in un righello metallico, e dunque il passo è $d = 1 mm$), possiamo ricavare $\lambda$ dalla relazione:

$$ n\lambda = d(cos\theta_i-cos\theta_r) $$

con $l = 344 cm $ (distanza sorgente-schermo) e $\theta = arctg\frac{h}{l}$

Per considerare l'errore che commettiamo su $l$, dunque, la funzione che ci interessa è:
$$\lambda\sage{lamb}$$
Il cui differenziale è:
$$\nabla \lambda \sage{dif}$$

Dunque, l'incertezza su $\lambda$ è data dalla relazione
$$\sigma_\lambda = \sage{errscal}$$

\begin{sagesilent}
subd = errscal.subs(d=0.1,h=30,l=elle,sigma_h=0.3,sigma_k=0.3,sigma_l=3)
sublambda = lamb.subs(d=0.1,h=30,l=elle)

lambdas = [abs((sublambda.subs(h=int(dato['h']), k=dato['k'], n=int(dato['n']))*10^7)).n(digits=4) for dato in dati]
errorsarr = np.array([(subd.subs(n=int(dato['n']), k=dato['k'])*10^7).n(digits=4) for dato in dati], dtype='f4')
print errorsarr

dati = ml.rec_append_fields(dati, "lambda", lambdas)
dati = ml.rec_append_fields(dati, "sigma_lambda", errorsarr)

meanlambda = np.average(dati['lambda'], weights=1./(dati['sigma_lambda']**2))
errtot = 1/sum(1/dati['sigma_lambda']**2)

def stampa_dati(wa, header):
  s = r"\begin{tabular}{c*{" + "%d" % (len(wa.dtype)+1)
  s += r"}{|c}}"
  s += "%s \\\\" % (header)
  s += r"\midrule"
  for i in range(0, len(wa)):
    a = ["%.4G" % x for x in wa[i]]
    s += "%s \\\\" % join(a, "&")
  s += r"\end{tabular}"
  return s
\end{sagesilent}

Sostituendo:
\begin{center}
\sagestr{stampa_dati(dati, "n & k (cm) &h (cm)& $\lambda$ (nm) & $\sigma_\lambda$ ")}
\end{center}
Graficati:
\begin{sagesilent}
import matplotlib.pyplot as plt
plt.clf()
plt.errorbar(dati['k'], dati['lambda'], dati['sigma_lambda'], fmt="ro")
plt.grid(True)
plt.ylabel(r"$\lambda$ (nm)")
plt.xlabel("k (cm)")
plt.savefig("grafici/interfer-lambda-con-err.png", dpi=300)
\end{sagesilent}

\includegraphics[scale=0.75]{grafici/interfer-lambda-con-err.png}

Valor medio: $\sage{int(meanlambda)}\pm\sage{int(errtot)}$ nm
