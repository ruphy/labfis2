\begin{sagesilent}
import matplotlib.pyplot as plt
import numpy as np
import scipy.optimize as opt
\end{sagesilent}


\chapter{Microonde}

\section{Riflessione}

Intendiamo verificare la legge di Cartesio utillizando una lastra di metallo come superficie rifelttente, posizionandola su un supporto magnetico sul goniometro.

\begin{equation}
sin(\theta_{incidente}) = sin(\theta_{rifratto})
\end{equation}

\begin{sagesilent}
dati = np.recfromcsv('dati/MICROONDE/cartesio.csv')
\end{sagesilent}

\begin{center}
\sagestr{stampa_dati(dati, r'$\theta_{inc}$ (rad) & $\theta_{rif}$ (rad) ' )}
\end{center}


\section{Misura della lunghezza d'onda}

\begin{sagesilent}
dati = np.recfromcsv('dati/MICROONDE/lambda.csv')

#Calcolo la lunghezza d'onda:

d = np.array(dati['distanza'])
d.astype(float)
n = np.array(dati['enne'])
n.astype(float)
lam = 2*d/n
sigmad = 0.5
errlam = 2/n*sigmad

lmedia = np.average(lam)
confronto = abs(2.85-lmedia)

dati = ml.rec_append_fields(dati, 'lungonda', lam)

\end{sagesilent}

Se si pongono trasmettitore e ricevitore ad una distanza di $\frac{n \lambda}{2}$, si può osservare che essi generano onde stazionarie. Infatti le antenne delle due apparecchiature riflettono parzialmente le onde che ricevono, e se li si mette in una condizione tale per cui le onde riflesse hanno la stessa fase delle onde incidenti si osservano dei massimi, viceversa, se le fasi sono antagoniste, si avrà una condizione di nodo.
Ricaviamo la lunghezza d'onda posizionando emettitore e ricevitore ai capi del metro, e muovendo il ricevitore lungo questo osserviamo  il passaggio alterno per massimi e minimi di intensità. Ricaviamo $\lambda$ dalla nota formula per le onde stazionarie:
\begin{equation}
d = n\frac{\lambda}{2}
\label{d}
\end{equation}


\begin{center}
\sagestr{stampa_dati(dati, r'$d$ (cm) & $n$ (rad) & $\lambda$ (cm)' )}
\end{center}

Otteniamo una lambda media: $$ \sage{lmedia} \pm \sage{errlam}$$

%Dal confronto con la $\lambda$ (2.85 cm) teorica otteniamo: $\lambda_{teo} - \lambda_{cal} || = \sage{confronto} < 2 \sigma ( = 2 \sage{errlam} $


\section{Rifrazione attraverso un prisma}

In questa parte dell'esperienza si verifica la legge di snell:
\begin{equation}
\frac{sin(\theta_{i})}{sin(\theta_{r})} = \frac{n_2}{n_1}
\end{equation}
A tal fine posizioniamo una pedana di polistirolo sul goniometro, e vi poniamo sopra un prisma, sempre di polistirolo, contenente "styrene pellets". Si verifica la legge per vari massimi di intensità (Nota: $n_1$ = 1).

%Dati???

\section{Polarizzazione}

\begin{sagesilent}
#Calcolo dell'intensità con la legge di Malus

dati = np.recfromcsv('dati/MICROONDE/malus.csv')

izero = 8.75
gamma=dati['gamma']*3.14/180
iteorico = izero*(cos(gamma))**2

dati = ml.rec_append_fields(dati, 'iteo', iteorico)

\end{sagesilent}


Le microonde uscenti dal trasmettitore sono polarizzate linearmente, per cui soddisfano alla legge di Malus che lega l'intensità all'angolo tra la normale al ricevitore e la direzione dell'onda. Fissata una distanza, posizioniamo emettitore e ricevitore uno difronte all'altro e misuriamo la diminuzione d'intensità relativa al variare dell'angolo rispetto all'orizzontale di emettitore-ricevitore. Di seguito i dati letti sul multimetro a confronto con l'intensità teorica data dalla legge di Malus:

\begin{equation}
I = I_{0} cos^2 \gamma
\end{equation}

\begin{center}
%\sagestr{stampa_dati(dati, r'$I$ (Volt) & $\gamma$ (rad) & $I_{teorico}$ (Volt) ' )}
\end{center}


\section{Interferenza da doppia fenditura}
\begin{sagesilent}
dfen = np.recfromcsv("dati/microonde-doppiafen.csv")
dfen.sort()
var('x,a,w,t')
model(x, a, w, t) = a*sin(w*x+t)

def ff(x, a,w,t):
  f = fast_callable(model)
  return f(x,a,w,t)

#print dfen['angolo']
pars, pcov = opt.curve_fit(ff, np.array(dfen['angolo']), np.array(dfen['volt']), p0=[500., 50., 1.])

print pcov

print pars
plt.clf()
xin = np.arange(min(dfen['angolo']), 220, 1)
print "now"
yin = ff(xin, pars[0], pars[1], pars[2])
plt.plot(xin, yin)
plt.plot(dfen['angolo'], dfen['volt'], 'o--')

plt.savefig("grafici/microonde-doppiafend.png")
\end{sagesilent}


Utilizzando un supporto magnetico sul goniometro, montiamo tre lastre di metallo in modo da avere due
fenditure di circa 1,5 cm. Spostando il ricevitore si troviamo i massimi e i minimi di interferenza al fine di verificare la
loro posizione rispetto alla previsione teorica: $d \sin(\theta) = n \lambda$ (per i massimi, con d distanza delle fenditure, $\theta$ angolo
rispetto alla perpendicolare, n numero intero).

\includegraphics[scale=0.75]{grafici/microonde-doppiafend.png}

\section{Specchio di Lloyd}

\begin{sagesilent}
#Calcolo della lunghezza d'onda

dati = np.recfromcsv('dati/MICROONDE/lloyd.csv')

d = dati['delta']
enne = dati['n']

lam = 2*d/enne


dati = ml.rec_append_fields(dati, 'lung_onda', lam)

\end{sagesilent}

Posizioniamo emettitore e ricevitore a distanza di un circa un metro, uno dinnanzi all'altro. Su un secondo metro, perpendicolare alla direzione emettitore-ricevitore e passante per il goniometro, montiamo una lastra di metallo che funge da specchio: in tal modo generiamo un'interferenza dovuta alla differenza di cammino delle onde (una parte arriva direttamente al ricevitore, mentre altre onde percorrono il cammino dall'emettitore allo specchio e dallo specchio al ricevitore). Cerchiamo il minimo di intensità con lo specchio a distanza minima dal goniometro e raccogliamo i massimi di intensità al variare della distanza, per determinare $\lambda$ come in \ref{d}

\begin{center}
%\sagestr{stampa_dati(dati, r'$d$ (cm) & $n$ & $\lambda$ ' )}
\end{center}

\section{Diffrazione di Bragg}

\begin{sagesilent}
#Grafico dell'intensità in funzione dei gradi

dati = np.recfromcsv('dati/MICROONDE/bragg.csv')
plt.clf()
plt.plot(dati['theta'], dati['volt'],'ro')

plt.savefig("grafici/microonde-bragg.png")

\end{sagesilent}


Nell'ultima parte dell'esperimento ci proponiamo di verificare la legge di Bragg, 
\begin{equation}
n \lambda = 2 d \sin(\theta)
\end{equation}
la quale esprime analiticamente i fenomeni di interferenza causati dalla riflessione di onde su piani paralleli di un reticolo cristallino. Per riprodurre il fenomeno, facciamo incidere le microonde su un cubo di polistirolo su cui sono inserite su spaziature regolari delle sferette di acciaio.

Nella formula:\\
$\theta$ è l'angolo che il fascio incidente forma col piano cristallino\\
$\lambda$ è la lunghezza d'onda della radiazione\\
$d$ è la distanza tra due piani adiacenti\\
$n$ indica l'ordine della diffrazione (tipicamente solo quello per n=1 è apprezzabile).\\

La formula si spiega in maniera analitica considerando una differenza di cammino ottico pari a $2d\sin(\theta)$.


\includegraphics[scale=0.75]{grafici/microonde-bragg.png}

\section{Analisi dati}

\section{Allegato: dati}
\begin{sagesilent}
def stampa_dati(wa, header):
  s = r"\begin{tabular}{c*{" + "%d" % (len(wa.dtype)-1)
  s += r"}{|c}}"
  s += "%s \\\\" % (header)
  s += r"\midrule"
  for i in range(0, len(wa)):
    a = ["%s" %x for x in d[i]]
    s += "%s \\\\" % join(a, "&")
  s += r"\end{tabular}"
  return s
\end{sagesilent}

\begin{center}

% \sagestr{stampa_dati(d, r"Frequenza (Hz) & I (mA) & Valore mu & $V_{rms}$ (V)")}
\end{center}

