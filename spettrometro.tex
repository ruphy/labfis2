%INIZIALIZZAZIONE e LIBRERIE
\begin{sagesilent}
#Importo librerie necessarie
import numpy as np
from scipy import odr
import matplotlib.pyplot as plt
import matplotlib.mlab as ml

#classica funzione chiquadrato, non cancellare
def chiquad(xdata, ydata, yfunc, ysigma = (), param=()):
    yteo = yfunc(xdata, *param)
    ddof = len(xdata) - 1 - len(param)
    if (np.isscalar(ysigma)):
        ys = np.ones_like(xdata)*ysigma
    else:
        ys = ysigma
 
    if (len(ys)):
        return sum(((ydata-yteo)/ys)**2)
    else:
        return sum((ydata-yteo)**2/yteo)

        
#Funzione stampadati
def stampa_dati(datiarr, header):
  s = r"\begin{tabular}{c*{" + "%d" % (len(datiarr.dtype)-1)
  s += r"}{|c}}"
  s += "%s \\\\" % (header)
  s += r"\midrule"
  for i in range(0, len(datiarr)):
    a = ["%.3G" %x for x in datiarr[i]]
    s += "%s \\\\" % join(a, "&")
  s += r"\end{tabular}"
  return s
        
\end{sagesilent}



\chapter{ Spettrometro (O3)}


\section*{Reticolo}

\subparagraph{Introduzione}

Nella prima parte dell'esperienza misuriamo le lunghezze d'onda corrispondenti a differenti righe spettrali emesse da una sorgente di gas eccitato, attraverso l'uso di un reticolo a diffrazione. \\
Il reticolo a diffrazione consiste in una lastra di vetro su cui sono tracciate delle linee parallele molto sottili. La distanza tra le fenditure dà il passo del reticolo.\\
La luce proveniente dalla sorgente viene fatta passare prima attraverso una fenditura, e successivamente attraverso una lente convergente, il cui piano focale coincide proprio con il piano su cui giace la fenditura: tale sistema prende il nome di collimatore. I raggi paralleli uscenti dalla lente intercettano il reticolo, che attraverso una procedura sperimentale viene posto ortogonalmente al fascio, e da lì vengono deviati di un angolo $\theta$. Un telescopio montato su un piatto rotante permette di osservare le righe d'emissione del gas, ruotando opportunamente la piattaforma secondo la direzione $\theta_{\lambda} $ data dalla legge:
\begin{equation}
sin \theta_{\lambda} = m \frac{\lambda}{d}
\label{eq:theta}
\end{equation}


Da tale legge, dunque possiamo risalire alla lunghezza d'onda, attraverso la lettura dell'angolo di cui si è ruotato il telescopio (la piattaforma è dotata di un goniometro).

\subparagraph{Passo del reticolo}

Prima di iniziare la raccolta dati è necessario posizionare il reticolo in modo tale che risulti perpendicolare al fascio di luce incidente. A tal fine misuriamo gli angoli di deviazione per massimi dello stesso ordine, e lo zero corrisponde alla direzione della luce non deviata. Otteniamo $\theta_{0} = 283 3' $ \\
Per stabilire il passo del reticolo utilizziamo una sorgente di cui è nota la lunghezza d'onda (lampada a sodio: $\lambda_{Na} = 589.3 $ nm ), e giriamo la \ref{eq:theta} per ricavare d. In particolare utilizziamo una versione della \ref{eq:theta} che diminuisca l'errore sperimentale a cui sono soggette le misure, in cui a $\theta$ si sostituisce la media dei due angoli destro e sinistro $\theta_{dx} $ e $\theta_{sx}$.

\begin{equation}
d = \frac{n \lambda}{sin(\frac{\theta_{dx}-\theta_{sx}}{2})}
\end{equation}

L'errore su d si ottiene dalla propagazione degli errori associati alla misura degli angoli. Si ha dunque:

\begin{sagesilent}


dati = np.recfromcsv('dati/spettro-reticolo.csv')
var('n')
var('dtheta', latex_name=r'\Delta\theta')
var('lam', latex_name=r'\lambda')
var('sigma_dtheta', latex_name=r'\sigma_{\Delta\theta}')
d(dtheta)=n*lam/(sin(dtheta))
derror = (d.diff())*sigma_dtheta
sigmadth=2*0.0087

dfasterror = fast_float(derror)
#print derror.subs(lam=1,n=2,dtheta=1.).n()

# dati['deltatheta'] = dati['deltatheta']*10^(-3)

def ret_error(dato):
  deltheta = dato['deltatheta']
  res = derror(dtheta=deltheta).subs(lam=dato['lambda']*10^-9, n=int(dato['n']),
                    sigma_dtheta=sigmadth)
  return abs(res[0])
  
errorsarr = np.array([ret_error(dato) for dato in dati ])

dcalc = np.array([float(d.subs(lam=dato['lambda']*10^-9,n=int(dato['n']),dtheta=dato['deltatheta']).n(digits=2) ) for dato in dati ])
dati = ml.rec_append_fields(dati, "d", dcalc*10^6)
dati = ml.rec_append_fields(dati, "error_d", errorsarr*10^6)

\end{sagesilent}
$$d:\sage{d}$$
Derivata e moltiplicata per $\sigma_{\Delta\theta}$, otteniamo $\sigma_d$:
$$\sigma_d: \sage{derror}$$
Per i nostri dati:
\begin{center}
\sagestr{stampa_dati(dati, r'n & $\lambda$ (nm) & $\Delta\theta$ (rad) & d ($\mu$m)& $\sigma_d$ ($\mu$m)' )} 
\end{center}

\begin{sagesilent}
media = np.average(dati['d'], weights=1./dati['error_d']**2)
errmedia = 1./np.sqrt(sum(1./dati['error_d']**2))
\end{sagesilent}

Otteniamo $d=\sage{round(media, 3)}\pm\sage{errmedia.n(digits=2)}$ $\mu$m.




\section*{Prisma}


Nella seconda parte dell'esperienza intendiamo verificare la legge di Cauchy:
\begin{equation}
n = a + \frac{b}{\lambda^2}
\label{Cauchy}
\end{equation}
che lega l'indice di rifrazione di un materiale alla lunghezza d'onda della luce incidente, nel caso dell'indice di rifrazione di un prisma di vetro.
I coefficienti a e b che figurano in \ref{Cauchy} vengono determinati con il metodo dei minimi quadrati. \\
Pertanto misuriamo n corrispondente alle differenti righe spettrali del mercurio (utilizziamo una lampada Hg come sorgente) attraverso la legge:
\begin{equation}
n = \frac{sin(\frac{\alpha + \delta}{2})}{sin(\frac{\alpha}{2})}
\label{n}
\end{equation}
in cui $\alpha$ è l'angolo al vertice del prisma, di $60°$, e $\delta$ l'angolo di deviazione minima. Nota: al denominatore del membro di destra si è trascurato di mettere $n_{aria}=1$.\\

Con tale procedura otteniamo i seguenti valori:

\begin{sagesilent}
#Fit per ricavare i parametri della relazione di Cauchy

dati = np.recfromcsv('dati/spettro-cauchy.csv')

l = 1./(dati['l']**2)
n = dati['n']

var('x,P')

def func(P,x):
    return P[0]*x+P[1]

odrfit, chi = fit_chiquad(l, n, func, init_guess=[1.,1.], ysigma=l*0.01)

xin = np.arange(0.9*min(l),1.3*max(l),2*max(l)/10) 
yin = func(odrfit.beta, xin)

plt.clf()
plt.plot(l,n,'ro')
plt.plot(xin,yin,'--')
plt.grid(True,which="both")
plt.savefig("grafici/spettro-cauchy.png", dpi=300)
\end{sagesilent}

\begin{center}
 \includegraphics[scale=0.75]{grafici/spettro-cauchy.png}
\end{center}


$$a = \sage{odrfit.beta[1]} \pm \sage{odrfit.sd_beta[1]}$$
$$b = \sage{odrfit.beta[0]} \pm \sage{odrfit.sd_beta[0]}$$
$$ \chi^ 2 = \sage{chi}$$

Dati raccolti

\begin{sagesilent}


import numpy as np
import matplotlib.mlab as ml

dati = np.recfromcsv('dati/spettro-cauchy.csv')

var('n, alpha')
var('delta', latex_name=r'\delta')
var('sigma_delta', latex_name=r'\sigma_{\delta}')

d(delta)=sin((alpha+delta)/2)/sin(alpha/2)

derror = (d.diff())*sigma_delta
sigmadel=0.0087
alpha_nostro=1.047


def ret_error(dato):
  res = derror(delta=dato['delta']).subs(alpha=alpha_nostro, n=int(dato['n']),
                    sigma_delta=sigmadel)
  return abs(res[0])
  
errorsarr = np.array([ret_error(dato) for dato in dati ])

dcalc = np.array([float(d(delta=dato['delta']).subs(alpha=alpha_nostro, n=int(dato['n'])).n(digits=2) ) for dato in dati ])

print dati
print dcalc
print errorsarr
dati = ml.rec_append_fields(dati, "error_d", errorsarr)

def stampa_dati(datiarr, header):
  s = r"\begin{tabular}{c*{" + "%d" % (len(datiarr.dtype)-1)
  s += r"}{|c}}"
  s += "%s \\\\" % (header)
  s += r"\midrule"
  for i in range(0, len(datiarr)):
    a = ["%.3G" %x for x in datiarr[i]]
    s += "%s \\\\" % join(a, "&")
  s += r"\end{tabular}"
  return s
\end{sagesilent}

\begin{center}
\sagestr{stampa_dati(dati, r"$\lambda$ (nm) & $n$ & $\delta$ & $\sigma_n$")}
\end{center}


In cui l'errore su $n$ è dato dall'errore su $\delta = 0.0087$ rad tramite la funzione:

$$\sigma_n(\delta): \sage{derror}$$


Dal fit otteniamo i seguenti parametri:

\begin{sagesilent}
 #Fit per ricavare i parametri della relazione di Cauchy


dati = np.recfromcsv('dati/spettro-cauchy.csv')

l = 1./(dati['l']**2)
n = dati['n']

var('x,P')

#Questa funzione serve solo al chiquadro e per il grafico
def fun(x,a,b):
    return a*x+b


#Questa funzione serve per il fit fatto da odr
def func(P,x):
    return P[0]*x+P[1]


mymodel = odr.Model(func)

mydata = odr.RealData(l,n)

myodr = odr.ODR(mydata, mymodel, beta0=[1.,1.],  maxit=5000)
myout = myodr.run()

a = myout.beta[0]
b = myout.beta[1]
chiquadrato = chiquad(np.array(l), n, fun, ysigma=errorsarr, param=[a,b]) 

   
plt.clf()
plt.plot(l,n,'ro')
xin = np.arange(0.9*min(l),1.3*max(l),2*max(l)/10) 
yin = fun(xin,myout.beta[0],myout.beta[1])
plt.plot(xin,yin,'--')

plt.errorbar(l,n,errorsarr,np.zeros_like(errorsarr),'bo')

plt.grid(True,which="both")
plt.savefig("grafici/spettro-cauchy.png", dpi=300)
 

\end{sagesilent}


$$a = \sage{myout.beta[1]} \pm \sage{myout.sd_beta[1]}$$
$$b = \sage{myout.beta[0]} \pm \sage{myout.sd_beta[0]}$$

Effettuamo il test del ${\chi}^2$ per verificare la bontà delle ipotesi, ed otteniamo:

$$ \chi^ 2 = \sage{chiquadrato}$$



\begin{center}
\includegraphics[scale=0.75]{grafici/spettro-cauchy.png}
\end{center}




\section*{Determinazionedi una sorgente ignota}
Infine utilizziamo i procedimenti e i valori stimati nelle prime parti dell'esperienza  per determinare, mediante lo spettro, il gas corripondente alla sorgente ignota.\\

Calcoliamo n dalla \ref{n}, ed utilizziamo tali valori per ricavare le lunghezze d'onda dalla \ref{Cauchy}, dove adesso sono noti i parametri a e b:
\begin{equation}
\lambda = \sqrt{\frac{b}{n-a}}
\end{equation}


%Elaborazione dati:


\begin{sagesilent}


dati = np.recfromcsv('dati/spettro-neon.csv')

var('alpha')
var('delta', latex_name=r'\delta')
var('sigma_delta', latex_name=r'\sigma_{\delta}')


n(delta) = sin((alpha+delta)/2)/sin(alpha/2)

nerror = (n.diff())*sigma_delta
sigmadel=0.0087
alpha_nostro=1.047
coeff1=9333.5
coeff2=1.5919
sigma_a = 0.000766
sigma_b = 164.4


def ret_error(dato):
  res = nerror(delta=dato['delta']).subs(alpha=alpha_nostro, sigma_delta=sigmadel)
  return abs(res[0])
  
errorsarr = np.array([ret_error(dato) for dato in dati ])

ncalc = np.array([float(n(delta=dato['delta']).subs(alpha=alpha_nostro).n(digits=2) ) for dato in dati ])

#Determinazione delle lambda incongite

var('a', 'b')
var('enne', latex_name='n')

l(enne)=sqrt(b/(enne-a))

lcalc= np.array([float(l(enne=dato,digits=4).subs(b=coeff1, a=coeff2)  ) for dato in ncalc ])

#Propagazione errori

var('sigma_a', 'sigma_b')

lerror(a,b)=sqrt(b/(enne-a))
dif = lerror.diff()
s_l = sqrt(dif[0]*sigma_a +dif[1]*sigma_b)

#show(s_l)

#dif=vector([lerror.diff(a), lerror.diff(b)])

#def errscal(dato):
 # errs = vector([sigma_a, sigma_b])
 # slam = vector([0, 0])
 # diffVal = dif(a=coeff1, b=coeff2, digits=4).subs(enne=dato)
  
 # for i in range(0, len(dif)):
 #     slam[i] = (diffVal[i]*errs[i])
    
 # return abs(slam)

#print 'ncalc', ncalc
#def errscal(dato):
    #dato = s_l.subs(sigma_a=sigma_a, sigma_b=sigma_b, a=a, b=b, n=dato['n'])
#    return dato
#errorl = np.array([float(errscal(trick) ) for trick in ncalc ])

#print dati
#print ncalc
#print errorsarr
#print lcalc
#print errorl
dati = ml.rec_append_fields(dati, "n", ncalc)
dati = ml.rec_append_fields(dati, "error_n", errorsarr)
dati = ml.rec_append_fields(dati, "l", lcalc)
#dati = ml.rec_append_fields(dati, "error_l", errorl)

\end{sagesilent}

\begin{center}
\sagestr{stampa_dati(dati, r"  $\delta$ & $n$ & $\sigma_n$ & $\lambda$  ") }
\end{center}


Dal confronto con i valori tabulati otteniamo che l'accordo migliore è con lo spettro del neon. Infatti, come si può osservare:


