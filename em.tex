

\chapter{Dati geometrici}
\begin{center}
\begin{tabular}{|c|c|c|c|c|c|c|c|}
\midrule
\textit{Solenoide} & $n$ & $R_{int}$ & $R_{ext}$ & $R_{medio}$ & $l_{circonferenza}$ & $h$ & $A_{sezione}$  \\
		   & 322 & 0,05	 & 0,0937  & 0,07185 & 0,451446864 & 0,0225 &	0,00098325 \\
 \midrule
\end{tabular}
\end{center}


\begin{center}
\begin{tabular}{|c|c|c|c|c|}
\midrule
\textit{Capacità} & $R_{int}$ (cm) & $R_{ext}$ (cm) & $lunghezza$ (cm) & $spessore piastre$ (cm)\\
   & 5.89 & 3.546  & 40  & 0.31 \\
%?? nel file ods ci sono altre misure per gli stessi campi!
 \midrule
\end{tabular}
\end{center}

\subsubsection{Misura della lunghezza d'onda del laser}

Misuriamo c dalla relazione:
\begin{equation}
 c=\frac{1}{sqrt{LC}}
\end{equation}
pertanto misuriamo la frequenza di risonanza di un cicuito RLC, per differenti configurazioni del circuito.

\subparagraph*{C ed L}
Nella prima parte dell'esperienza abbiamo verificato che i valori di capacità ed induttanza calcolati a partire dai parametri geometrici corrispondessero effettivamente a quelli misurati in laboratorio. 
Per tanto, in entrambi i circuiti, calcoliamo algebricamente i valori di V e $V_[0]$ per cui $t = \tau$. Inseriamo tali valori e misuriamo i tempo di scarica. Dalla stima del tempo caratteristico ricaviamo L e C. La capacità risulta in accordo, ed è stimata intorno ai $4.38$ nF, l'induttanza invece risulta:

\begin{center}
\begin{tabular}{c c}
$L_{geometrico}$  & 0.000293\\
$L_{misurato}$ & 0.000325\\
\end{tabular}
\end{center}

Visto che è presente una discrepanza tra i valori misurati e quelli geometrici, dovuto a capacità parassite, d'ora in avanti lavoreremo con una induttanza equivalente (modificando un parametro geometrico per ottenere l'accordo): $N_{eq} = 339$


\begin{comment}
Costruiamo un circuito RLC. Nota: il condensatore risente molto delle fluttuazioni, per cui lo colleghiamo a terra, per ridurle il più possibile.

Frequenze di risonanza intorno ai 500 kH (nota: è un minimo -> potenziale ai capi di LC)
\end{comment}

\subparagraph*{L=40 cm}
\begin{sagesilent}
#Fit per stimare i parametri -> Frequenza di risonanza: -b/2a

dati = np.recfromcsv('dati/em-parabola1.csv')


\end{sagesilent}



\begin{comment}
\begin{center}
\sagestr{stampa_dati(dati, r'$\omega$ (rad) & $\V_{out}$ (nm) & $\sigma_{V_{out}}$ (nm)' )}
\end{center}
\end{comment}

\subparagraph*{L=38 cm}

\subparagraph*{L=34 cm}

\subparagraph*{L=32 cm}

\subparagraph*{L=28 cm}

\chapter{Analisi con capacità parassite in serie}
\chapter{Analisi con capacità parassite in parallelo}


