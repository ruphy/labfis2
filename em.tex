

\chapter{Dati geometrici}
\begin{center}
\begin{tabular}{|c|c|c|c|c|c|c|c|}
\midrule
\textit{Solenoide} & $n$ & $R_{int}$ & $R_{ext}$ & $R_{medio}$ & $l_{circonferenza}$ & $h$ & $A_{sezione}$  \\
		   & 322 & 0,05	 & 0,0937  & 0,07185 & 0,451446864 & 0,0225 &	0,00098325 \\
 \midrule
\end{tabular}
\end{center}


\begin{center}
\begin{tabular}{|c|c|c|c|c|}
\midrule
\textit{Capacità} & $R_{int}$ (cm) & $R_{ext}$ (cm) & $lunghezza$ (cm) & $spessore piastre$ (cm)\\
   & 5.89 & 3.546  & 40  & 0.31 \\
%?? nel file ods ci sono altre misure per gli stessi campi!
 \midrule
\end{tabular}
\end{center}

\subsubsection{Misura della lunghezza d'onda del laser}

Misuriamo c dalla relazione:
\begin{equation}
 c=\frac{1}{sqrt{LC}}
\end{equation}
pertanto misuriamo la frequenza di risonanza di un cricuito RLC, per differenti configurazioni.


\begin{comment}
Costruiamo un circuito RLC. Nota: il condensatore risente molto delle fluttuazioni, per cui lo colleghiamo a terra, per ridurle il più possibile.

Frequenze di risonanza intorno ai 500 kH (nota: è un minimo -> potenziale ai capi di LC)
\end{comment}

\subparagraph*{Configurazione 1}
\begin{sagesilent}
#Fit per stimare i parametri -> Frequenza di risonanza: -b/2a

dati = np.recfromcsv('dati/em-parabola1.csv')


\end{sagesilent}



\begin{comment}
\begin{center}
\sagestr{stampa_dati(dati, r'$\omega$ (rad) & $\V_{out}$ (nm) & $\sigma_{V_{out}}$ (nm)' )}
\end{center}
\end{comment}



\chapter{Analisi con capacità parassite in serie}
\chapter{Analisi con capacità parassite in parallelo}


